\chapter*{Streszczenie}
Niniejsza praca opisuje proces powstawania projektu pod tytułem: ``Interaktywna wizualizacja stron Wikipedii w jaskini rzeczywistości wirtualnej''. Jest inspirowany Projektem Opte, którego twórcy podjęli próbę zwizualizowania całego internetu, skupia się jednak na mniejszym, bardziej osiągalnym celu: pokazywaniu związków i połączeń między artykułami zamieszczonymi w Wikipedii. 

Po wstępie zostały wyszczególnione wymagania oraz scenariusze użycia tak jak zostały one zdefiniowane przed rozpoczęciem prac nad projektem. Następnie opisany jest proces wytwarzania, poczynając od pozyskania i przetworzenia danych. Pobrane zrzuty z baz Wikipedii są poddawane ciągowi przekształceń usuwających niepotrzebne informacje oraz kompresujących pozostałe w przygotowaniu do wygodnego użycia przez główną aplikację. Cały proces jest zamknięty w prostym programie z interfejsem użytkownika pozwalającym na wybranie typu przetwarzanej Wikipedii oraz monitorowanie postępu czasochłonnego tworzenia plików danych.

Opis aplikacji obejmuje zastosowane techniki reprezentacji wizualnej, z uwzględnieniem kroków powziętych w celu optymalizacji symulacji, szczegółowy opis jej działania oraz dostępnych funkcji. Są to między innymi dwa tryby poruszania się po wizualizacji, konsola operatora z możliwością wyszukiwania oraz automatyczne odtwarzanie przygotowanych wcześniej akcji. Wyjaśniony jest też proces dostosowania działania projektu do środowiska Laboratorium Zanurzonej Wizualizacji Przestrzennej. Dodatkowo rozwinięty zostaje koncept linii czasu, która została opisana w specyfikacji wymagań, jednak z przyczyn technicznych nie mogła zostać dodana.

Rozdział ``Eksperyment'' dokumentuje użycie aplikacji w docelowym środowisku dużej jaskini Laboratorium Zanurzonej Wizualizacji Przestrzennej. W podsumowaniu znajduje się przegląd całości projektu. Zamieszczone zostało tam też zestawienie pierwotnych założeń z końcowym produktem oraz wnioski autorów.\\

\noindent\textbf{Słowa kluczowe:} 

\noindent\textbf{Dziedzina nauki i techniki, zgodnie z wymogami OECD:} Nauki inżynierskie i techniczne, Elektrotechnika, elektronika, inżynieria informatyczna, Sprzęt komputerowy i architektura komputerów
