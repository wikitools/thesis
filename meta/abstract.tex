\chapter*{Abstract}
This paper describes the process of creating the project titled: ``Interactive visualization of Wikipedia pages in the Immersive 3D Visualization Lab''. It is inspired by the Opte Project, whose creators have attempted to visualize the entire Internet. The project however focuses on a smaller, more achievable goal: showing relations and connections between Wikipedia articles.

After the introduction, the requirements and use cases have been listed as they had been defined before any work on the project begun. Next, the production process is described, starting with the data acquisition and processing. Downloaded dumps from Wikipedia databases are subjected to a series of transformations removing unnecessary information and compressing the data in preparation for use by the main application. The whole process is enclosed in a simple program with a user interface that allows to easily select the type of Wikipedia to be processed and to monitor the progress of this time-consuming task.

The application description includes the techniques used in visual representation as well as the steps taken to optimize the simulation and a detailed description of its inner workings and available functions. These include, among others, two modes of navigating the visualization, the operator's console with the ability to search and automatically play sets of actions prepared earlier. The process of adapting the project to the environment of the Immersive 3D Visualization Lab is also explained. In addition, the concept of a timeline, which was described in the requirements but for technical reasons could not be added is discussed in greater detail.

The ``Experiment'' chapter documents the application being used in its target environment: large cave in the Immersive 3D Visualization Lab. The summary contains an overview of the entire project, comparison of the original assumptions with the final product as well as the authors' conclusions.\\

\noindent\textbf{Keywords:} Wikipedia, articles, connections, data, visualization, cave, Immersive 3D Visualization Lab, route, console, integration, requirements
