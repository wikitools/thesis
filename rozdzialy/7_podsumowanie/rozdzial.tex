\begin{chapter}{Podsumowanie}
	\newcommand{\chapterPath}{rozdzialy/7_podsumowanie}
	\label{ch:podsumowanie}

	\noindent Podział zadań w projekcie oraz przy pisaniu pracy:
	\begin{enumerate}[label=\textbullet]
		\item \textbf{Stanisław Góra:} Podstawa aplikacji, integracja ze środowiskiem jaskini, nadzór nad całością
		\item \textbf{Mikołaj Mirko:} Interfejs użytkownika oraz wygląd aplikacji oraz programu do tworzenia danych
		\item \textbf{Jan Kruczyński:} Przetwarzanie danych Wikipedii na potrzeby aplikacji, ładowanie węzłów
		\item \textbf{Mateusz Janicki:} Dodatkowe funkcje aplikacji: historia, trasy, wyszukiwanie
	\end{enumerate}
	
	Głównym celem pracy inżynierskiej było zaproponowanie rozwiązania nietypowego przedstawienia treści dobrze znanego serwisu Wikipedia i zainteresowania nim potencjalnego użytkownika. Uwaga odbiorcy skierowana jest na strukturę Wikipedii i połączenia między artykułami i kategoriami, a nie ich treść.

	Powstała aplikacja spełnia postawione na początku założenie. Wizualizacja grafu połączeń w przestrzeni pozwala zauważyć istniejące związki między stronami, które trudno zauważyć przeglądając serwis w przeglądarce internetowej. Aplikacja dołączy również do biblioteki materiałów dostępnych do uruchomienia w LZWP i pozwoli zaprezentować różne możliwości jaskini odwiedzającym ją gościom. Większość dostępnych aplikacji odzwierciedla świat rzeczywisty, nasza aplikacja pozwala wprowadzić użytkownika w zupełnie nowy, wygenerowany dynamicznie świat.

	Projektowanie aplikacji na środowisko jaskini wirtualnej rzeczywistości wiąże się z wieloma trudnościami. Projekt od rozpoczęcia pracy nad nim uległ wielu zmianom, co łatwo zauważyć walidując specyfikację napisaną na samym początku z uzyskanym produktem końcowym. Bazowe założenia pozostały jednak bez zmian.

	Wśród zmian w wymaganiach funkcjonalnych (Sekcja \ref{sec:wymagania-funkcjonalne}) znajduje się widok szczegółowy (podgląd części treści w odrębnym widoku aplikacji). Zrezygnowano z tej funkcjonalności ze względu na aktualny brak dostępu do Internetu w średniej i dużej jaskini. Widoki grafu (swobodnego poruszania się) oraz węzła (podgląd połączeń) zostały zaimplementowane bez zmian (Sekcja \ref{sec:tryby-widoki}). Funkcjonalność automatycznego i losowego poruszania się po węzłach grafu została zamieniona na możliwość odtwarzania wcześniej przygotowanych tras (Sekcja \ref{sec:history}). Celem tej opcji było ułatwienie prezentowania związków między artykułami, lecz w przypadku losowego podróżowania trudno odróżnić interesujące połączenia od pozostałych.

	Dużą zmianą wprowadzoną w specyfikacji funkcjonalności była rezygnacja z linii czasu. Ze względu na niewystarczające rozeznanie oraz błędne założenia dotyczące oferowanych przez wybrane źródło danych informacji niemożliwe było zgromadzenie potrzebnych do tej funkcjonalności danych. Szczegółowy opis tej problemowej funkcjonalności został opisany w sekcji \ref{sec:linia-czasu}. Względem spisanych wymagań, zmianom uległ również sposób wyświetlania informacji o węzłach, infografiki instruktarzowe oraz dodatkowe elementy (Sekcja \ref{sec:elementy}). Nagłówek został uszczegółowiony, a pozostałe elementy (takie jak na przykład siatka wspomagająca) dodane w celu zwiększenia użyteczności aplikacji.

	W związku z opisanymi modyfikacjami schemat kontrolera uległ zmianie. Zredukowana została liczba kontrolerów do jednego, a jego przyciski zostały rozłożone trochę inaczej - tak, aby korzystanie z kontrolera było wygodniejsze. Nowy schemat znajduje się na Rysunku \ref{fig:flystick_new_controls}. To jak i inne parametry aplikacji zostały, w ramach testów, dostosowane do potrzeb opisanych w wymaganiach jakościowych (Sekcja \ref{sec:wymagania-jakosciowe}). Ograniczenie ilości funkcjonalności pozwoliło na skupienie się na pozostałych elementach aplikacji oraz dopracowanie ich.

	Nieprzewidzianym przez nas aspektem pracy stało się wydobycie potrzebnych danych. W pierwotnych założeniach sprowadzało się do pobrania i prostego przetworzenia informacji. Z czasem pojawiały się jednak kolejne problemy (dostępność danych, ich format i rozmiar, oferowane informacje o pojedynczym artykule itp.). Odpowiedzią na skomplikowany proces pozyskiwania i przetwarzania danych jest dodatkowe narzędzie WikiGraph Parser (opisane w rozdziale \ref{ch:dane}). Zostało stworzone w celu ułatwienia generowania plików wejściowych do głównej aplikacji oraz zebraniu algorytmów przetwarzających w jednym miejscu.

	Podczas wykonywania projektu rozwiązane zostało wiele napotkanych problemów. Generowanie plików wykorzystywanych przez aplikację zostało przyspieszone (pomimo, często bardzo dużych, rozmiarów plików), a ich czytanie uproszczone wykorzystując sprytnie rozplanowane binarnych plików. Rozmieszczenie i wizualizacja grafu, a także zastosowane elementy interfejsu sprawnie przekazują niezbędne informacje użytkownikowi, zachowując przy tym prosty, ale zrozumiały wygląd. Integracja elementów jaskini, takich jak okulary i kontrolery, została uproszczona poprzez własny system zarządzania różnymi środowiskami oraz przypisanymi do przycisków i joysticków akcjami. Rozróżnienie środowiska PC do testów na lokalnych komputerach przy pomocy klawiatury i myszki oraz środowiska CAVE do wyświetlania aplikacji w jaskini i sterowania przy pomocy kontrolerów niebywale zoptymalizowało nasza pracę nad aplikacją. Brak klawiatury został częściowo rozwiązany poprzez stworzoną konsolę operatora.

	Aplikacja z pewnością może być rozwijana w przyszłości o dodatkowe funkcjonalności, zwiększając jej wartość jako narzędzie analityczne. Uwzględniając w wizualizacji dodatkowe dane, takie jak popularność lub rozmiar treści artykułu, można skupić się na innych zależnościach wewnątrz struktur Wikipedii. Rozwinięcie możliwości sterowania aplikacją o system gestów lub komendy głosowe powinno podwyższyć stopień imersji i ułatwić użytkownikowi odnalezienie się w przestrzeni grafu.

\end{chapter}
