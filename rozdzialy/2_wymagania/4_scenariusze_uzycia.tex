\section{Scenariusze użycia}
\begin{enumerate}
\item Przejście z węzła ``Politechnika Gdańska'' do węzła ``Gdańsk''

	W celu wędrówki po powiązanych artykułach należy najpierw znaleźć interesujący nasz węzeł - ``Politechnika Gdańska''. Do ruchu w widoku grafu użytkownik wykorzystuje joystick głównego kontrolera. Po odszukaniu węzła użytkownik wskazuje na niego kontrolerem i wybiera go za pomocą przycisku spustu. Po wybraniu wokół użytkownika pokażą się pierwsze alfabetycznie artykuły powiązane z aktualnie wybranym. W celu znalezienia artykułu ``Gdańsk'' należy przesunąć przedziały wyświetlanych powiązań, naciskając przycisk na kontrolerze, aż go znajdziemy. Po jego wskazaniu i wybraniu przeszliśmy do interesującego nas artykułu.
	
\item Zmiana daty wyświetlanego grafu na 15-ty tydzień  2015 roku z daty 2-gi tydzień 2019 roku

    Aby zmienić datę, należy początkowo wybrać interesującą nas jednostkę czasu. W tym wypadku najlepiej zacząć od zmiany roku - wybieramy jednostkę za pomocą joysticka (przechylając go na lewo lub prawo) na kontrolerze pomocniczym. Również za pomocą joysticka (poruszając go w przód i w tył) cofamy się cztery lata w czasie. Po cofnięciu się w latach, możemy wybrać dokładniejszą jednostkę, jaką jest tydzień. Czas przewijamy trzynaście razy do przodu. Po zaakceptowaniu zmian następuje przebudowanie grafu. Po ukończeniu operacji widoczny jest graf zawierający artykuły istniejące w wybranym momencie czasowym.
	
\item Wyświetlanie statystyk o artykule ``Netflix''

	Do przeprowadzenia tej operacji wymagane jest połączenie z Internetem. Proces wyświetlania statystyk rozpoczynamy od znalezienia i wybrania węzła. Użytkownik porusza się w widoku grafu za pomocą joysticka głównego kontrolera, wskazuje na węzeł i wybiera go używając przycisk spustu. Następnie wystarczy przejść do widoku szczegółowego za pomocą przycisku \textit{Details} na kontrolerze. Wokół użytkownika pojawi się okno podglądu szczegółowych informacji na temat artykułu, wraz z jego krótką treścią i obrazem.
	
\item Przejście do artykułu należącego do tej samej kategorii co artykuł ``Jan Matejko’’

    Po znalezieniu artykułu ``Jan Matejko’’ w przestrzeni, należy go wskazać i wybrać za pomocą przycisku spustu. Wokół użytkownika pokażą się połączenia z innymi artykułami. W celu znalezienia kategorii do której należy wybrany artykuł, należy najpierw zmienić typ wyświetlanych połączeń na połączenia prowadzące do niego od innych węzłów, celując joystickiem w węzeł na którym stoi użytkownik i naciskając spust. Połączenia dookoła użytkownika zmienią się i należy zlokalizować dowolną kategorię, która będzie przedstawiona za pomocą innego kształtu geometrycznego. Jeżeli nie wyświetlają się żadne kategorie, należy przewinąć wyświetlane połączenia joystickiem w górę i w dół, aż jakaś kategoria się wyświetli. Należy w nią wycelować kontrolerem i nacisnąć spust, a stanie się ona aktualnie wybranym węzłem. Spomiędzy aktualnie wyświetlanych połączeń należy wybrać artykuł o tytule innym niż ``Jan Matejko’’ i tak samo jak poprzednio wycelować w niego i nacisnąć spust. Jest to artykuł, który należy do tej samej kategorii co artykuł ``Jan Matejko’’.
	
\item Wyświetlenie pomocy przy sterowaniu i zmiana układu sterowania

	Aby wyświetlić pomoc dotyczącą sterowania aplikacją, należy wcisnąć przycisk \textit{Help} na kontrolerze pomocniczym. Ukaże się wtedy szczegółowa infografika z możliwościami sterowania, jakie oferują kontrolery. Jeśli nie odpowiada nam układ sterowania z powodu lewo- lub praworęczności, można go zmienić wciskając przycisk spustu podczas wyświetlania pomocy, określając tym samym nowy główny kontroler. Po opuszczeniu pomocy (klikając ponownie przycisk \textit{Help}) zmiana układu kontrolerów zostanie zastosowana.
\end{enumerate}
