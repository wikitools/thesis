\section{Scenariusze użycia}
\begin{enumerate}
\item Przejście z węzła ``Politechnika Gdańska'' do węzła ``Gdańsk''

	W celu wędrówki po powiązanych artykułach należy najpierw znaleźć interesujący nasz węzeł - ``Politechnika Gdańska''. Do ruchu w widoku grafu użytkownik wykorzystuje joystick głównego kontrolera. Po odszukaniu węzła użytkownik wskazuje na niego kontrolerem i wybiera go za pomocą przycisku spustu. Po wybraniu wokół użytkownika pokażą się pierwsze alfabetycznie artykuły powiązane z aktualnie wybranym. W celu znalezienia artykułu ``Gdańsk'' należy przesunąć przedziały wyświetlanych powiązań naciskając przycisk na kontrolerze, aż go znajdziemy. Po jego wskazaniu i wybraniu przeszliśmy do interesującego nas artykułu.
	
\item Zmiana daty wyświetlanego grafu na 15-ty tydzień  2015 roku z daty 2-gi tydzień 2019 roku

	Aby zmienić datę należy początkowo wybrać interesującą nas jednostkę. W tym wypadku najlepiej zacząć od zmiany roku - wybieramy jednostkę za pomocą joysticka na kontrolerze pomocniczym. Również za pomocą joysticka cofamy się cztery lata w czasie. Po cofnięciu się w latach, możemy wybrać dokładniejszą jednostkę, jaką jest tydzień. Czas przewijamy trzynaście razy do przodu. Po zaakceptowaniu zmian następuje przebudowanie grafu. Po ukończeniu operacji widoczny jest graf zawierający artykuły istniejące w wybranym momencie czasowym.
	
\item Wyświetlanie statystyk o artykule ``Netflix''

	Do przeprowadzenia tej operacji wymagane jest połączenie z Internetem. Proces wyświetlania statystyk rozpoczynamy od znalezienia i wybrania węzła. Użytkownik porusza się w widoku grafu za pomocą joysticka głównego kontrolera, wskazuje na węzeł i wybiera go używając przycisk spustu. Następnie wystarczy przejść do widoku szczegółowego za pomocą odpowiedniego przycisku na kontrolerze. Wokół użytkownika pojawi się okno podglądu szczegółowych informacji na temat artykułu wraz z jego krótką treścią i obrazem.
	
\item Wyświetlanie kategorii, do których należy artykuł ``Jan Matejko''

	Po znalezieniu artykułu w przestrzeni, po której poruszamy się za pomocą joysticka głównego kontrolera, należy go wskazać i wybrać za pomocą przycisku spustu. Wokół użytkownika pokażą się połączenia z innymi artykułami. W celu wyświetlenia do jakich kategorii należy artykuł wystarczy przełączyć tryb wyświetlania powiązań na tryb wyświetlania kategorii  za pomocą odpowiedniego przycisku. Ukaże się wtedy alfabetycznie pierwszy zestaw kategorii, do których należy artykuł. Przedziały można przewijać za pomocą przycisku.
	
\item Wyświetlenie pomocy przy sterowaniu i zmiana układu sterowania

	Aby wyświetlić pomoc dotyczącą sterowania aplikacją należy wcisnąć odpowiedni przycisk na kontrolerze. Ukaże się wtedy szczegółowa infografika z możliwościami sterowania, jakie oferują kontrolery. Jeśli nie odpowiada nam układ sterowania z powodu lewo- lub praworęczności, można ją zmienić wciskając przycisk wyświetlany na infografice. Następnie, po wskazaniu i wybraniu odpowiedniej opcji, układ zostanie zmieniony.
\end{enumerate}
