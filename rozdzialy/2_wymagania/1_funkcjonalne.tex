\section{Wymagania funkcjonalne}
\label{sec:wymagania-funkcjonalne}

W treści każdego artykułu na Wikipedii znajdują się odnośniki prowadzące do innych artykułów. Na podstawie tej własności można stworzyć sieć połączeń pomiędzy wszystkimi stronami. Dodatkowo, każdy artykuł należy do kategorii określającej jego tematykę. Kategorie łączą się także pomiędzy sobą, określając swoje kategorie nadrzędne i podrzędne.

Aplikacja opiera się na grafie stworzonym z węzłów reprezentujących artykuły i kategorie oraz połączeń określających związki pomiędzy nimi. Taka struktura jest przedstawiana użytkownikowi w kilku trybach, tak aby umożliwić interakcję z grafem na różne sposoby. Istnieją trzy widoki przeglądania danych: widok grafu, widok węzła oraz widok treści.

Widok grafu (rysunek \ref{fig:prototyp_widok_grafu}) jest głównym widokiem aplikacji. W przestrzeni rozmieszczone są węzły w postaci punktów. Użytkownik może się swobodnie poruszać (``latać'') we wszystkich kierunkach. Połączenia pomiędzy punktami nie są wyświetlane w celu zachowania przejrzystości wyświetlanych informacji. Punkty można wskazywać, a następnie wybierać. Po wykonaniu tej czynności otwierany jest widok węzła.

\img{\chapterPath/img/prototyp_widok_grafu.jpg}{Prototyp widoku grafu}{prototyp_widok_grafu}{0.8}

Widok węzła umieszcza wybrany artykuł lub kategorię na dolnym ekranie. Połączenia wychodzące z tego węzła rozłożone są na ścianach bocznych. Aby ułatwić wskazywanie połączonych węzłów, będą one pokazywane w formie otaczających użytkownika kopii punktów, zawieszonych na krzywych biegnących od wybranego węzła do oryginalnego punktu. Użytkownik może zmienić typ wyświetlanych połączeń, z tych które prowadzą od wybranego węzła do innych, na połączenia które prowadzą do wybranego węzła od innych. Powyższy tryb został zilustrowany na rysunku \ref{fig:prototyp_widok_wezla}.

\img{\chapterPath/img/prototyp_widok_wezla.jpg}{Prototyp widoku węzła (przypadek widoku powiązań artykułów)}{prototyp_widok_wezla}{0.8}
Aby nie przytłaczać użytkownika zbyt dużą ilością informacji, wyświetlana jednocześnie liczba połączeń jest ograniczona, a utworzone przedziały można przewijać. Po wybraniu połączonego węzła przenosimy się do nowo wybranego węzła. Możliwe jest cofanie się w historii przeglądania, a także ponowne wykonywanie cofniętej operacji. W tle widoczne są węzły z widoku grafu, jednak wyłączona jest możliwość ich wybierania w celu ułatwienia wskazywania powiązań. Nazwa wybranego węzła wyświetlana jest ponad poziomem głowy użytkownika zgodnie z kierunkiem w który on patrzy. Istnieje opcja automatycznego poruszania się pomiędzy węzłami, która ułatwia prezentację aplikacji. Polega ona na przechodzeniu do losowo wybranych węzłów powiązanych z aktualnie wybranym do momentu, w którym zostanie ona ręcznie zatrzymana. W każdym momencie możliwe jest opuszczenie widoku węzła i przejście do widoku grafu. Użytkownik w takim przypadku zostaje ustawiony obok opuszczonego węzła. 

Do widoku treści można przejść z poziomu widoku węzła. Wyświetlane są w nim statystyki i informacje ogólne na temat aktualnie wybranej strony na Wikipedii. Dostępny jest również podgląd fragmentu treści w przypadku artykułu.

W widoku grafu i widoku węzła widoczna jest także oś czasu. Pokazuje ona datę, z której wyświetlany jest stan grafu. Zmienia ona swoje położenie na podstawie ruchów głowy użytkownika tak, aby zawsze znajdowała się nad nim. Zmiana czasu na osi powoduje pokazywanie tylko takich węzłów, które istniały w wybranym momencie (wraz z ich połączeniami).

Aplikacja wymaga zastosowania dwóch kontrolerów ze względu na dużą liczbę interakcji dostępną dla użytkownika. Dla zapewnienia wyższej jakości obsługi istnieje opcja wyboru układu przycisków - praworęczny oraz leworęczny. Wybór ten możliwy jest przy starcie aplikacji. W momencie wyświetlania okna wyboru widoczna jest także infografika z informacjami o sterowaniu. Można otworzyć ją także w trakcie użytkowania aplikacji. Z pozycji infografiki można także zmienić tryb sterowania za pomocą wyświetlanego przycisku.

Za pomocą głównego kontrolera (rysunek \ref{fig:schemat_kontroler_glowny}) można wskazywać i wybierać węzły zarówno w widoku grafu, jak i w widoku węzła. Za pomocą joysticka w każdym widoku można dokonywać ruchu kamerą. W widoku grafu służy on także do swobodnego poruszania się, a w widoku węzła do przesuwania grup połączeń. Cztery pozostałe przyciski służą do resetowania widoku do stanu początkowego (przycisk \textit{Home}), cofania się w historii przeglądania (przycisk \textit{Back}), ponowienia cofniętej historii przeglądania (przycisk \textit{Forward}) oraz opuszczenia widoku węzła i przejście do widoku grafu (przycisk Exit).

\img{\chapterPath/img/schemat_kontroler_glowny.png}{Schemat kontrolera głównego}{schemat_kontroler_glowny}{0.8}

Kontroler pomocniczy (rysunek \ref{fig:schemat_kontroler_pomocniczy}) wykorzystuje joystick oraz trzy przyciski. Joystick służy do manipulowania osią czasu. Za jego pomocą można zmieniać datę. Pierwszy przycisk służy do automatycznej wycieczki po węzłach (przycisk \textit{Auto}, ponowne jego naciśnięcie zatrzymuje wycieczkę), drugi - do wyświetlenia treści i statystyk wybranego artykułu (przycisk \textit{Details}). Trzeci przycisk (przycisk \textit{Help}) powoduje wyświetlenie pomocy z infografiką o sterowaniu, która jest dostępna w każdym widoku. Przyciski pierwszy i drugi są dostępne tylko w trybie widoku węzła.

\img{\chapterPath/img/schemat_kontroler_pomocniczy.png}{Schemat kontrolera pomocniczego}{schemat_kontroler_pomocniczy}{0.8}
