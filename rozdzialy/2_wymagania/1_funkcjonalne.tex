\section{Wymagania funkcjonalne}
\label{sec:wymagania-funkcjonalne}

Budowa aplikacji musi być zgodna z wymaganiami jaskini rzeczywistości wirtualnej. Aplikacja będzie uruchamiać się jednocześnie na wielu komputerach, dlatego dane muszą się aktualizować dostatecznie szybko, aby zachować płynność i zgodność wyświetlanych obrazów.

Aplikacja musi posiadać kilka widoków przeglądania danych: widok grafu, widok treści oraz widok węzła, który dzieli się na widok artykułu i widok kategorii.

Widok grafu (Rysunek \ref{fig:prototyp_widok_grafu}) jest głównym widokiem aplikacji. Użytkownik może się w nim swobodnie poruszać (``latać'') we wszystkich kierunkach. Widoczne są artykuły i kategorie umieszczone w przestrzeni w formie ``punktów''. Obiekty są ze sobą ściśle powiązane, tworzą strukturę grafu (możliwe są jednak węzły nie posiadające żadnych połączeń). Same połączenia w widoku nie są wyświetlane w celu zachowania przejrzystości wyświetlanych informacji. Artykuły i kategorie można wskazywać, a następnie wybierać. Po wykonaniu tej czynności przechodzimy do widoku artykułu lub widoku kategorii - w zależności od typu wybranego węzła. Oba te widoki nazywamy w ogólności widokiem węzła.

\img{\chapterPath/img/prototyp_widok_grafu.jpg}{Prototyp widoku grafu}{prototyp_widok_grafu}{0.8}

Widok węzła przedstawia wybrany węzeł na dolnym ekranie oraz połączenia z innymi węzłami. Aby ułatwić wskazywanie połączeń, połączone węzły będą pokazywane w formie otaczających użytkownika kopii ``punktów'' i będą zawieszone na krzywych biegnących od wybranego węzła z dolnego ekranu do oryginalnego ``punktów''. W zależności od tego, czy wybrany jest artykuł czy kategoria, mamy dwa tryby wyświetlania powiązań. Jeśli wybraliśmy artykuł, w pierwszym trybie wyświetlane są powiązane artykuły (w sensie wystąpienia linku do tego artykułu w wybranym artykule), a w drugim najniższe (najbardziej szczegółowe) kategorie, do których należy wybrany artykuł. Pierwszy tryb został zilustrownay na Rysunku \ref{fig:prototyp_widok_wezla}.

\img{\chapterPath/img/prototyp_widok_wezla.jpg}{Prototyp widoku węzła (przypadek widoku powiązań artykułów)}{prototyp_widok_wezla}{0.8}

W przypadku wybrania kategorii wyświetlane są albo artykuły i kategorie, które bezpośrednio należą do wybranej kategorii, albo najniższe kategorie, do których należy wybrana kategoria. Aby nie przytłaczać użytkownika zbyt dużą ilością informacji, wyświetlana jednocześnie liczba połączeń jest ograniczona, utworzone przedziały będzie można przewijać. Po wybraniu połączonego węzła przenosimy się do widoku nowo wybranego węzła. Możliwe jest cofanie się w historii przeglądania, a także ponowne wykonywanie cofniętej operacji. W tle widoczne są węzły z widoku grafu, jednak wyłączona jest możliwość ich wybierania w celu ułatwienia wskazywania powiązań. Nazwa wybranego węzła wyświetlana jest na górze i zmienia swoją pozycję poziomą, śledząc ruch głowy użytkownika. Istnieje opcja automatycznego poruszania się po powiązaniach artykułów i kategorii, która ułatwia prezentację aplikacji. Polega ona na przechodzeniu do losowo wybranych węzłów powiązanych z aktualnie wybranym do momentu, w którym zostanie ona ręcznie zatrzymana. W każdym momencie możliwe jest opuszczenie widoku węzła i przejście do widoku grafu. Zostajemy wtedy ustawieni obok węzła, z którego widoku wyszliśmy. Z każdego widoku jest możliwe także przejście do widoku grafu z kamerą wycentrowaną na główną kategorię całego grafu. Do widoku szczegółowego można przejść z poziomu widoku węzła. Wyświetlane są w nim statystyki dotyczące węzła i informacje ogólne na temat strony.

W trybie grafu i trybie węzła na górnej części widoku widoczna jest także oś czasu. Pokazuje ona czas, z którego wyświetlany jest stan grafu i wybraną jednostkę czasu. Zmienia ona swoje położenie na podstawie ruchów głowy użytkownika tak, aby zawsze znajdowała się nad nim. Zmiana czasu na osi powoduje pokazywanie tylko takich węzłów, które istniały w wybranym czasie (wraz z ich połączeniami).

Aplikacja wymaga zastosowania dwóch kontrolerów ze względu na dużą liczbę interakcji dostępną dla użytkownika. Aby zapewnić wyższą jakość obsługi zaimplementowana zostanie opcja wyboru układu przycisków - praworęczny oraz leworęczny. Wybór ten będzie możliwy przy starcie aplikacji. W momencie wyświetlania okna wyboru widoczna będzie także infografika z informacjami o sterowaniu. Będzie można  otworzyć ją także w trakcie przeglądania danych. Z pozycji infografiki można także zmienić tryb sterowania za pomocą wyświetlanego przycisku.

Za pomocą głównego kontrolera (Rysunek \ref{fig:schemat_kontroler_glowny}) będzie można wskazywać i wybierać węzły zarówno w widoku grafu, jak i w widoku węzła. Za pomocą joysticka w każdym widoku można dokonywać ruchu kamerą. W widoku grafu służy on także do swobodnego poruszania się, a w widoku węzła do przesuwania grup połączeń. Cztery pozostałe przyciski służą do resetowania widoku do stanu początkowego (przycisk Home), cofania się w historii przeglądania (przycisk Back), ponowienia cofniętej historii przeglądania (przycisk Forward) oraz opuszczenia widoku węzła i przejście do widoku grafu (przycisk Exit).

\img{\chapterPath/img/schemat_kontroler_glowny.png}{Schemat kontrolera głównego}{schemat_kontroler_glowny}{0.8}

Kontroler pomocniczy (Rysunek \ref{fig:schemat_kontroler_pomocniczy}) wykorzystuje joystick oraz trzy przyciski. Joystick służy do manipulowania osią czasu. Za jego pomocą można zmieniać jednostkę czasu oraz dokonywać zmiany czasu. Pierwszy przycisk służy do automatycznej wycieczki po węzłach (przycisk Auto, ponowne jego naciśnięcie zatrzymuje wycieczkę) oraz do wyświetlenia szczegółów i statystyk wybranego artykułu (przycisk Details). Oba przyciski dostępne są tylko w trybie widoku węzła. Trzeci przycisk (przycisk Help) powoduje wyświetlenie pomocy z infografiką o sterowaniu, która jest dostępna w każdym widoku.

\img{\chapterPath/img/schemat_kontroler_pomocniczy.png}{Schemat kontrolera pomocniczego}{schemat_kontroler_pomocniczy}{0.8}
