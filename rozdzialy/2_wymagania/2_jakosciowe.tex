\section{Wymagania jakościowe}
\begin{enumerate}[label=\textbullet]
\item \textbf{Niezawodność}

Aplikacja nie może posiadać żadnych błędów, które powodowałyby zaburzenie immersji. Należą do nich wszelkie błędy graficzne, błędy w interfejsie, jak również nieprawidłowe umiejscowienie węzłów i połączeń w przestrzeni. Aplikacja musi mieć także dobrze zaprojektowaną i przemyślaną strukturę grafu, ponieważ stanowi ona rdzeń programu. Jakiekolwiek błędy z nią związane powodowałyby niezdolność do korzystania z aplikacji i niepowodzenie całego projektu. W przypadku wykrycia takich błędów powinny one uzyskać najwyższy priorytet i być naprawione w następnej wersji.

\item \textbf{Użyteczność}

Aplikacja musi być estetyczna i wygodna w użyciu. Ważne jest zastosowanie nowoczesnych animacji i innowacyjnego interfejsu, pozwalającego zanurzyć się w wizualizacji. Użytkownik ma się poczuć, jakby naprawdę podróżował po stworzonym świecie. Interfejs musi wykorzystywać zalety jaskini rzeczywistości wirtualnej - możliwość ruchu użytkownika i śledzenie ruchów głowy. Sterowanie aplikacją, mimo dużej ilości interakcji, powinno być intuicyjne, a w razie problemów musi istnieć pomoc dla użytkownika.

\item \textbf{Czas reakcji}

Aplikacja, ze względu na swoją specyfikę, musi być przyjemna w odbiorze dla użytkownika. Czas przejścia z widoku grafu do widoku węzła lub odwrotnie oraz czas podróży pomiędzy węzłami nie powinien trwać więcej niż 1,5 sekundy. Dopuszczalny jest długi czas uruchamiania aplikacji ze względu na wymóg zbudowania grafu z pliku, jednak nie powinien on trwać więcej niż 20 sekund, gdyż spowodowałoby to obniżenie zadowolenia użytkownika i spowolniłoby to prezentowanie możliwości jaskini odwiedzającym. 
\end{enumerate}

``Aby zapewnić widok dla wielu osób maksymalnie niezależny od pozycji okularów wiodących proponuję umiejscowić obiekty przede wszystkim w odległości odpowiadającej pozycji ścian jaskini.''
