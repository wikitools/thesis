\begin{chapter}{Eksperyment}
	\newcommand{\chapterPath}{rozdzialy/6_eksperyment}	
	\sectionauthor{Mateusz Janicki}
Aplikacja została przetestowana na dużej jaskini, po pomyślnym przejściu testów na małej jaskini. Początkowo występowały drobne problemy z synchronizacją, jednak zostały one przez nas rozwiązane. Dopiero na środowisku docelowym istnieje możliwość sprawdzenia poprawności działania efektu 3D przy użyciu okularów zamieszczonych na Rysunku \ref{fig:okulary} oraz intensywniejszej synchronizacji między komputerami. 

\img{\chapterPath/img/okulary-min.JPG}{Okulary umożliwiające widzenie w 3D}{okulary}{.8}

Do testów aplikacji wykorzystane zostały scenariusze użycia opisane w sekcji 2.4. Ostatecznie tylko trzy z pięciu opisanych scenariuszy mogły zostać zrealizowane, ponieważ pozostałe korzystają z niezaimplementowanych i odrzuconych funkcjonalności. Ostatni może zostać zrealizowany tylko częściowo, ponieważ ostatecznie wykorzystywany jest tylko jeden kontroler, więc implementacja systemu zmiany układu sterowania była bezcelowa.

\begin{enumerate}[label=\textbullet]
	\item \textbf{Przejście z węzła “Politechnika Gdańska” do węzła ``Gdańsk''} 
	
Aplikacja została uruchomiona na dużej jaskini. Po włączeniu ukazała się przestrzeń informująca o podstawach sterowania. W celu przejścia do widoku grafu naciśnięty został przycisk 4 na kontrolerze. Spowodowało to odtworzenie krótkiej melodii i pojawienie się węzłów w przestrzeni. Z powodu ograniczonej liczby wyświetlanych węzłów w jednym momencie, znalezienie artykułu ``Politechnika Gdańska'' jest bardzo utrudnione. Znaleziony został on za pomocą konsoli operatora, co znacznie przyspieszyło cały proces. Użytkownik został przeniesiony w momencie wybrania nad węzeł, a na ścianach wokół niego pojawiły się połączenia do innych węzłów. Pojawił się nagłówek z informacją o aktualnie wybranym węźle, ilości połączeń i aktualnie wyświetlanym przedziale ze wszystkich połączeń. Jednocześnie wyświetlane ich było dwanaście, a węzeł “Gdańsk” nie znajdował się w pierwszym zestawie. Zmusiło to użytkownika do przewinięcia dalszych połączeń i szukania docelowego artykułu. Artykuł Gdańsk znaleziony został w siódmym zestawie połączeń. Użytkownik wskazał go kontrolerem, co spowodowało wyświetlenie się nazwy “Gdańsk” i liczby połączeń na nagłówku z informacjami. Odtworzona została także animacja wyświetlająca połączenia wskazywanego artykułu z innymi artykułami, a kolor węzła zmienił się na pomarańczowy. Użytkownik wcisnął przycisk spustu, co skutkowało zaznaczeniem węzła i wyświetlenie jego połączeń.
	\item \textbf{Przejście do artykułu należącego do tej samej kategorii co artykuł ``Jan Matejko''}
	
Po przejściu do widoku grafu ponownie otaczają nas jedynie losowe węzły, a prawdopodobieństwo pojawienia się wśród nich poszukiwanego artykułu jest bardzo niskie. Przed przeprowadzeniem została jednak przygotowana trasa zawierająca najbardziej znanych malarzy w Polsce, a więc wykorzystanie jej jest szybkim sposobem na znalezienie artykułu. Po otwarciu konsoli i przejściu do zakładki ``Routes'' został wciśnięty przycisk ``START'' odpowiadający trasie o nazwie ``Polscy malarze'', co spowodowało przeniesienie użytkownika na artykuł o nazwie ``Józef Chełmoński''. Ponadto zmianie uległa ikona stanu znadująca się pod nagłówkiem - dopisany do niej został napis ``AUTO'', który świadczy o tym, że jest aktualnie uruchomiona trasa.  Po odczekaniu wyznaczonego w konfiguracji czasu użytkownik jest przenoszony na następny artykuł o nazwie ``Aleksander Gierymski'', a po następnie na poszukiwany przez nas artykuł ``Jan Matejko''. W tym momencie naciśnięty został przycisk 4 powodujący przerwanie trasy oraz zniknięcie napisu ``AUTO'' z ikony stanu. Aby wyświetlić kategorie, należy skierować kontroler na wybrany węzeł i nacisnąć spust. Spowoduje to wyświetlenie zarówno kategorii, do których należy artykuł ``Jan Matejko'', jak i wyświetlenie artykułów, które mają odniesienie do naszego artykułu. Ponadto zmieni się ikona stanu symbolizująca skierowanie w grafie. Poszukiwane kategorie przedstawione są w formie fioletowych figur. Połączenia można przewijać za pomocą przycisków 2 i 3 na kontrolerze. Użytkownik odnalazł kategorię ``Polscy malarze'', wskazał ją i nacisnął spust, zostając przeniesionym na miejsce w grafie gdzie ta kategoria się znajdowała. Ponownie skierował kontroler na węzeł pod sobą i nacisnął spust by powrócić do typu wyświetlanych połączeń prowadzących do artykułów w tej kategorii. Następnie wskazał na artykuł ``Julian Cegliński'' i nacisnął spust.
	\item \textbf{Wyświetlenie pomocy przy sterowaniu i zmiana układu sterowania}
	
Inicjalnie pomoc dotycząca sterowania wyświetlana jest przy starcie aplikacji. Po przejściu do trybu grafu w każdym momencie możliwe jest jednak wyświetlenie pomocy za pomocą przycisku 1 na kontrolerze. Wychodzeniu i wchodzeniu do pomocy za każdym razem towarzyszy efekt dźwiękowy.
\end{enumerate}

Poza wymienionymi wyżej scenariuszami w trakcie eksperymentu sprawdzono również poprawność innych funkcji. Wielokrotnie wykonywano podstawowe akcje dostępne w aplikacji, takie jak pokazywanie pomocy przedstawione na Rysunku \ref{fig:pomoc}, czy wskazywanie artykułów i kategorii w trybie swobodnego lotu po grafie (Rysunek \ref{fig:swobodny_ruch}). 

\img{\chapterPath/img/pomoc-min.JPG}{Ekran pomocy zawierający informacje o sterowaniu}{pomoc}{.8}

\img{\chapterPath/img/swobodny_ruch-min.JPG}{Wskazywanie artykułu w trybie swobodnego lotu}{swobodny_ruch}{.8}

Ponadto poprawnie działa przechodzenie do trybu chodzenia po węzłach. Tryb ten został przedstawiony na Rysunku \ref{fig:artykul} dla zaznaczonego artykułu oraz na Rysunku \ref{fig:kategoria} dla zaznaczonej kategorii. 

\img{\chapterPath/img/artykul_wybrany-min.JPG}{Widok węzła dla wybranego artykułu}{artykul}{.8}

\img{\chapterPath/img/kategoria_wybrana_inne_kategorie-min.JPG}{Widok węzła dla wybranej kategorii}{kategoria}{.8}

Problematyczne okazało się wyświetlanie węzłów na krawędziach ścian jaskini. Ewentualne nieścisłości najbardziej widoczne są w trybie chodzenia po węzłach. Gdy przesuwamy kamerę za pomocą joysticka w lewo i prawo można zauważyć płynne przechodzenie grafik połączonych węzłów ze ściany na ścianę na Rysunku \ref{fig:zachodzenie}, co świadczy o poprawnym działaniu usuwaniu niewidocznych powierzchni (ang. frustum culling). 
Przetestowane zostało również wskazywanie (Rysunek \ref{fig:wskazywanie}) i wybieranie połączeń oraz zmiana trybu wyświetlania. Za każdym razem poprawnie wyświetlana jest ikona informująca o stanie połączeń. Aplikacja zachowuje się poprawnie przy cofaniu i ponawianiu akcji wykonanych przez użytkownika. 

\img{\chapterPath/img/kategoria_wybrana_artykuly-min.JPG}{Wyświetlanie prawidłowego kształtu dla węzła znajdującego się na dwóch ścianach jaskini}{zachodzenie}{.8}

\img{\chapterPath/img/kategoria_wskazywana_wybrana-min.JPG}{Wskazywanie kategorii w widoku węzła}{wskazywanie}{.8}

W momencie włączenia konsoli operatora, pojawia się ikona informująca użytkownika o otwartej konsoli (Rysunek \ref{fig:ikona_konsoli}). Operator może wyszukać dowolny węzeł, a wyniki wyszukiwania aktualizują się w momencie zmiany pola tekstowego (Rysunek \ref{fig:konsola}). Poprawnie działa również funkcjonalność związana z trasami - możliwe jest odtwarzanie i wyłączanie tras. Wszystkim akcjom użytkownika towarzyszy efekt dźwiękowy, co dodatkowo informuje użytkownika o odebraniu sygnału przez aplikację.

\img{\chapterPath/img/ikona_konsoli-min.JPG}{Ikona informująca o otwartej konsoli operatora}{ikona_konsoli}{.8}

\img{\chapterPath/img/konsola-min.JPG}{Konsola operatora na głównym komputerze w dużej jaskini}{konsola}{.8}

Po przeprowadzeniu eksperymentu można stwierdzić, że aplikacja, mimo braku kilku funkcjonalności zawartych w specyfikacji projektu, jest kompletna i użyteczna. Wybrane rozwiązania ułatwiające sterowanie idealnie spełniają swoje zadanie - konsola na głównym komputerze umożliwia przenoszenie na dowolne węzły oraz włączanie predefiniowanych tras. Immersja w aplikacji jest wysoka ze względu na dobrze działający efekt 3D, intuicyjny interfejs i klimatyczną muzykę. Całość powoduje przyjemny odbiór aplikacji i pokazuje nowy pomysł na wykorzystanie jaskini do wizualizacji tworów nie mających reprezentacji w rzeczywistości. 

\end{chapter}
