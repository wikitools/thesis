\section{Aplikacja Webverse}

Webverse \cite{Webverse} to aplikacja przeglądarkowa stworzona w listopadzie 2016 r. przez Owena Cornec, amerykańskiego artystę i dewelopera. W swojej pracy skupia się na wizualizacjach różnego typu danych. W przypadku tej aplikacji, zaprezentował on trójwymiarowy graf połączeń pomiędzy 200 tysiącami stron w sieci Internet. Projekt ten powstał w ramach obchodów dwudziestolecia istnienia Internet Archive\footnote{Internet Archive dostępne jest pod adresem https://archive.org/} – serwisu archiwizującego miliony stron internetowych oraz innych treści.

Aplikacja składa się z interfejsu (zawierającego wyszukiwarkę stron, ustawienia wyświetlania i listy najpopularniejszych lokalizacji) oraz przestrzeni z grafem (zrzut ekranu aplikacji na rysunku \ref{fig:webverse_screenshot}). Węzły grafu reprezentują domeny stron internetowych, a linie pojawiające się po zaznaczeniu węzła informują użytkownika o przynależności do jakiejś grupy. Autor aplikacji skupia się na wizualizacji skupisk (grup) stron, należących do tych samych lokalizacji geograficznych. Użytkownik jest więc w stanie w łatwy sposób porównać wielkość grupy stron pochodzących ze Stanów Zjednoczonych, z grupą reprezentującą Chiny.

Głównym problemem aplikacji jest uciążliwe poruszanie się w przestrzeni i nakładające się na~siebie punkty grafu, utrudniające interakcję. Dużą zaletą jest możliwość wyszukiwanie stron po nazwie oraz rejestrowanie historii poruszania się.

\img{\chapterPath/img/webverse.png}{Fragment grafu aplikacji Webverse \cite{Webverse}}{webverse_screenshot}{0.8}
