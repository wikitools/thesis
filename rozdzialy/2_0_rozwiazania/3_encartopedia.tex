\section{Pozostałe rozwiązania}

Wśród innych rozwiązań znaleźć można serwis Encartopedia\cite{Encartopedia}, który rozkłada artykuły na przestrzeni dwuwymiarowej. Autor aplikacji otwarcie przyznaje, że jego inspiracją było WikiGalaxy. Aplikacja skupia się jednak bardziej na przeglądaniu treści w klasyczny sposób, uatrakcyjniając go prostą mapą z punktami i oznaczonymi połączeniami (po wskazaniu kursorem na link wychodzący z artykułu – Rysunek \ref{fig:encartopedia_screenshot}). Źródłem danych, również w tym przypadku, jest pewien fragment anglojęzycznej Wikipedii.

\img{\chapterPath/img/wikigalaxy.png}{Serwis Encartopedia\cite{Encartopedia} z wybranym artykułem Mango i połączeniem do Banana}{encartopedia_screenshot}{0.8}

Na koniec warto wspomnieć o małym skrypcie zatytułowanym wiki-graph\cite{WikiGraphPython}, napisanym w języku Python. Generuje on rekurencyjnie dwuwymiarowy graf połączeń (wynik działania to plik obrazka). Możliwe jest zastosowanie skryptu do jakiejkolwiek Wikipedii, a dodatkowe parametry pozwalają dopracować wygląd grafu. Rozwiązanie te nie nadaje się jednak do tworzenia dużych struktur grafowych.