\section{Aplikacje WikiGalaxy i Wikiverse}

Owen Cornec (twórca Webverse \cite{Webverse}) jest również autorem kolejnych dwóch przytoczonych aplikacji: WikiGalaxy \cite{WikiGalaxy} (z roku 2014) oraz jej nowszej wersji Wikiverse\cite{Wikiverse} (z roku 2016). Oba projekty wizualizują artykuły oraz ich połączenia oferując dodatkowe narzędzia wspomagające pracę z aplikacją.

WikiGalaxy oferuje dwa widoki: mapy (widok z góry) oraz pierwszoosobowy (poruszanie się w~przestrzeni – widok zilustrowany na rysunku \ref{fig:wikigalaxy_screenshot}). Każdy ze 100 tysięcy artykułów reprezentowany jest jako punkt w przestrzeni. Kolor punktu określa jego przynależność do jednej z predefiniowanych kategorii. Po zaznaczeniu artykułu użytkownik jest w stanie przeglądać połączenia wychodzące z wybranego artykułu do innych artykułów. Aplikacja dostarcza (podobnie jak w przypadku Webverse) wyszukiwarkę i historię wykonanych akcji, ułatwiając nawigację po grafie. Na dzień 29 listopada 2019 r. aplikacja zdaje się mieć problemy z wczytywaniem tytułów artykułów.

\img{\chapterPath/img/wikigalaxy.png}{Widok pierwszoosobowy przestrzeni połączeń WikiGalaxy \cite{WikiGalaxy}}{wikigalaxy_screenshot}{0.8}

Wikiverse to sprawnie działający podgląd artykułów i ich połączeń w postaci grafu wzorowanego na galaktyce. Zawiera on 250 tysięcy artykułów rozmieszczonych w przestrzeni trójwymiarowej za pomocą specjalnego algorytmu fizycznego formułującego sferyczny rozkład punktów (rysunek \ref{fig:wikiverse_screenshot}). Historia odwiedzonych punktów znajduje się w panelu bocznym. Aplikacja nie posiada wyszukiwarki. Daje jednak możliwość szybkiego podglądu aktualnej treści artykułu wewnątrz aplikacji.

Niestety, w obu projektach wspierany jest tylko fragment anglojęzycznej Wikipedii. Grafy zostały wygenerowane odpowiednio w 2014 i 2016 roku. WikiGalaxy nie aktualizuje grafu od tego czasu, a~Wikiverse doczytuje brakujące dane tylko wtedy, kiedy są potrzebne.

\img{\chapterPath/img/wikiverse.png}{Wizualizacja połączeń w aplikacji Wikiverse \cite{Wikiverse}}{wikiverse_screenshot}{0.8}
