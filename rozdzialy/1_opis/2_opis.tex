\section{Opis projektu}
Tak jak w przypadku Opte Project, głównym zadaniem projektu jest przygotowanie grafu oraz jego wizualizacja. Nie jest to jednak graf połączeń sieci Internet, lecz sieć artykułów i kategorii internetowej encyklopedii Wikipedia. Tworzona jest ona na podstawie odnośników znajdujących się w treści artykułu, prowadzących do innych, tematycznie powiązanych, artykułów.

Nasza aplikacja oferuje nowy i innowacyjny sposób na przeglądanie Wikipedii: poprzez wizualizację połączeń pomiędzy artykułami. Dostępne są również narzędzia pozwalające na sprawne poruszanie się po grafie jak i dodatkowe funkcjonalności służące podstawowej analizie prezentowanych danych.

W celu pogłębienia immersji aplikacja została napisana na środowisko jaskini zanurzonej rzeczywistości wirtualnej znajdującej się w LZWP (Laboratorium Zanurzonej Wizualizacji Przestrzennej). Mając do dyspozycji widok ze wszystkich stron można przenieść użytkownika w dowolne miejsce skomplikowanego grafu co pozwoli mu przyjrzeć się połączeniom z bliska i lepiej zrozumieć strukturę Wikipedii.
