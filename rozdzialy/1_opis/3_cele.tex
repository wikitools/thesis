\section{Główne cele projektu}
Głównym celem projektu jest propozycja rozwiązania mającego zainteresować oraz inspirować odbiorców nietypowym przedstawieniem znanej powszechnie Wikipedii. Nietypowość polega głównie na~zmianie medium, przez które użytkownicy zwykle odbierają Wikipedię. Zamiast traktować ją jako zbiór artykułów, skupiamy się na uwidocznieniu związków między nimi, które mogłyby być trudne do~uchwycenia podczas przeglądania stron encyklopedii w przeglądarce internetowej.

Aplikacja wzbogaci nieustannie powiększający się zestaw materiałów dydaktycznych przygotowywanych w środowisku Laboratorium Zanurzonej Wizualizacji Przestrzennej na Politechnice Gdańskiej. Różnorodne aplikacje pozwalają lepiej zaprezentować możliwości rozwiązań wizualizacji przestrzennej gościom odwiedzającym LZWP.