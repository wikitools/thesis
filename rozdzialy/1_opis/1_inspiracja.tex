\section{Inspiracja}
\sectionauthor{Mikołaj Mirko}
W 2003 roku niejaki Barrett Lyon, informatyk oraz artysta, postawił sobie za cel stworzenie wiernego odwzorowania połączeń między komputerami w sieci Internet. Wizualizacja grafu miała być zrealizowana za pomocą kolorowych linii oraz punktów. W ten sposób w październiku 2003 roku powstał open source'owy Projekt Opte. Głównym celem tego projektu było zilustrowanie wciąż szybko rozwijającego się Internetu oraz wyróżnienie regionów, które w tamtych czasach doświadczały gwałtownego wzrostu łączności z Internetem.

Projekt szybko zyskał dużą popularność, a jeden z efektów końcowych można było zobaczyć na żywo w Muzeum Sztuki Nowoczesnej w Nowym Jorku. Przykładową wizualizację zaprezentowano na rysunku \ref{fig:opte-project}.

Twórca Opte Project w artykule dla Time \cite{OpteProject:Time} krótko podsumował swoją motywację:

\begin{center}
	\hyphenblockcquote{USenglish}{OpteProject:Time}{
		The Internet is really big, very connected and extremely complex. \linebreak
		It’s this whole world you can’t see. That’s the fun part of visualizing it.
	}
\end{center}

\img{\chapterPath/img/opte-project.jpg}{Jedna z wizualizacji stworzona przez Opte Project \cite{OpteProject:Time}}{opte-project}{0.7}
