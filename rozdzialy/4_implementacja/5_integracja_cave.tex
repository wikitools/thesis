\section{Integracja z jaskinią}
\sectionauthor{Stanisław Góra}
% LZWPLib
Przystosowanie aplikacji do działania w środowisku LZWP dzieli się na dwa podzadania: obsługa wejścia użytkownika oraz synchronizacja stanu aplikacji pomiędzy komputerami obsługującymi ściany jaskini. Reszta integracji jest zapewniana przez stworzoną przez pracowników LZWP bibliotekę.

% Input Module
\paragraph{Moduł wejścia}
W założeniu aplikacja ma wspierać dwie metody interakcji użytkownika, nazywane później środowiskami:
\begin{enumerate}[label=\textbullet]
	\item \textbf{PC} - gdzie użytkownik posługuje się klawiaturą i myszą, używane głównie podczas procesu wytwarzania aplikacji
	\item \textbf{Jaskini} - gdzie użytkownik nosi okulary i trzyma specjalny kontroler nazywany flystickiem
\end{enumerate}
Aby ułatwić sobie wytwarzanie głównej logiki aplikacji, wejście zostało odseparowane w osobny moduł który może być przed uruchomieniem ustawiony w jeden z dwóch trybów, odpowiadającym każdemu środowisku. Tryb ten definiuje jakiego rodzaju wejścia ma nasłuchiwać aplikacja. Moduł wejścia składa się z trzech mniejszych części:
\begin{enumerate}[label=\textbullet]
	\item \textbf{Definicje akcji użytkownika} - ustala wspólny interfejs dla każdej akcji użytkownika dostępnej w aplikacji. Są to między innymi: 
	\begin{enumerate}[label=\textbullet]
		\item Użycie przycisku klawiatury i myszy (PC) lub flysticka (Jaskinia)
		\item Przesunięcie myszy (PC) lub flysticka (Jaskinia)
	\end{enumerate}
	Oprócz tego jest odpowiedzialna za nasłuchiwanie wykonania przez użytkownika swojej akcji i odpowiednie powiadomienie aplikacji.
	
	\item \textbf{Definicje akcji aplikacji} - spis akcji wykonywanych w aplikacji. Są to między innymi: 
	\begin{enumerate}[label=\textbullet]
		\item Wybranie lub podświetlenie węzła
		\item Poruszanie się po grafie
		\item Zmiana zakresu wyświetlanych połączeń
	\end{enumerate}
	Każda akcja ma swój typ (opisany w poprzednim punkcie). Dostępny jest też interfejs dzięki któremu możliwe jest łatwe wybranie konkretnego przycisku (w każdym ze środowisk) który ma aktywować daną akcję \ref{fig:input-interface}.
	
	\item \textbf{Procesor akcji użytkownika} - odpowiada za połączenie modułu wejścia z pozostałą częścią aplikacji poprzez powiadamianie jej o każdej wykonanej przez użytkownika akcji po ewentualnym wstępnym przetworzeniu sygnału wejściowego.
\end{enumerate}

\img{\chapterPath/img/input-interface.png}{Fragment interfejsu wyboru przycisków dla środowiska PC}{input-interface}{.6}

% Networking
\paragraph{Synchronizacja stanu}
Jak już zostało wspomniane, w LZWP każda ściana jaskini sterowana jest przez osobny komputer. Tworzy to w wyniku środowisko aplikacji podobne do wieloosobowej gry w sieci LAN i wymusza synchronizację stanu z głównym komputerem odpowiedzialnym za przetwarzanie akcji użytkownika. W tym celu powstał w aplikacji moduł wysyłający zmiany w grafie komputerom do aktualizacji ekranów. 

Na opisanie zasługuje tu synchronizacja załadowanych lub usuniętych węzłów grafu. Mechanizm przesyłający wiadomości w sieci jest dosyć prosty i pozwala na dołączanie tylko najprostszych typów danych jak ciąg znaków, liczby i pozycja. Wysyłanie informacji o załadowaniu każdego węzła z osobna mogłoby zapełnić kanał komunikacyjny zwłaszcza przy starcie aplikacji kiedy ładowane są tysiące węzłów na raz. Został więc wprowadzony mechanizm którego celem jest zbieranie tych komunikatów i łączenie ich w większe paczki które po przesłaniu są z powrotem rozpakowywane i przetwarzane. Dzięki temu aplikacja zachowuje optymalne wykorzystanie łącza w jaskini.

% diagram tikz? input -> wstępne przetw i zapis serwera -> network sync -> aktualizacja stanu klientów -> aktualizacja ekranów | okrąg?
