\section{Ostateczny schemat interakcji}
\sectionauthor{Jan Kruczyński}
\label{sec:schemat_interakcji}
Po odrzuceniu funkcjonalności, takich jak widok szczegółowy i linia czasu, możliwa stała się obsługa aplikacji za pomocą tylko jednego kontrolera. Podział na kontroler główny (Rysunek \ref{fig:schemat_kontroler_glowny}) i pomocniczy (Rysunek \ref{fig:schemat_kontroler_pomocniczy}) został odrzucony, a wszystkie interakcje zostały umieszczone na jednym kontrolerze. Ten zabieg znacznie ułatwi użytkownikowi sterowanie aplikacją w jaskini. Nowy schemat sterowania został umieszczony na Rysunku \ref{fig:flystick_new_controls}.

\img{\chapterPath/img/nowy_schemat_kontrolera.png}{Aktualny schemat sterowania kontrolerem}{flystick_new_controls}{0.8}

Konsola operatora opisana w sekcji \ref{sec:konsola_operatora} uruchamiana jest na komputerze głównym klawiszem ``F1''. Możliwa jest obsługa za pomocą myszki i klawiatury lub wyłącznie klawiatury. Przełączanie pomiędzy zakładkami \textit{Search} i \textit{Routes} następuje po wciśnięciu klawisza ``Tab''. Poruszanie się po liście wyników wyszukiwania i liście dostępnych tras odbywa się za pomocą klawiszy strzałek, a potwierdzenie wyboru zaznaczonej pozycji to klawisz ``Enter''.