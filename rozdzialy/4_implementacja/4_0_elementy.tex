\section{Elementy zwiększające użyteczność aplikacji}
\sectionauthor{Mikołaj Mirko}
\label{sec:elementy}
Jednym z ważniejszych elementów znajdujących się w aplikacji jest podążający za wzrokiem nagłówek z informacjami o stanie grafu. Umiejscowiony jest on powyżej linii wzroku tak, aby nie przeszkadzał w interakcji z węzłami, a zarazem był wyraźnie widoczny z każdego miejsca oraz kąta patrzenia. Jego pozycja pionowa jest zaczepiona w stałym punkcie, a pozycja pozioma uzależniona od kierunku wyznaczanego przez główne okulary jaskini.

Na samej górze znajduje się informacja o tym, czy węzeł jest tylko wskazywany kontrolerem, czy jesteśmy w widoku grafu z aktualnie wybranym węzłem. Zaraz pod spodem widoczny jest tytuł artykułu lub kategorii. W trybie swobodnego latania nagłówek wyświetlany jest tylko w przypadku wskazywania węzłów. Rysunek \ref{fig:header1} nakreśla układ nagłówka po wskazaniu artykułu ``Matematyka''. Jeśli węzeł posiada jakieś połączenia, ich liczba wyświetlana jest pod tytułem.

\img{\chapterPath/img/header1.png}{Nagłówek po wskazaniu węzła}{header1}{.8}

Dla widoku węzła dostępna jest większa ilość informacji. Pod tytułem pojawia się podłużny wskaźnik przypominający poziomy pasek przewijania. Wraz z informacją tekstową, znajdującą się tuż pod nim, określa aktualnie wyświetlany wycinek listy połączeń. Łatwo zauważyć, które połączenia są wyświetlane wokół użytkownika oraz jaką część stanowią. Na Rysunku \ref{fig:header2} wybrany został artykuł zatytułowany ``Wstęga Möbiusa'' z 21 połączeniami wychodzącymi (aktualnie wyświetlane są połączenia od pierwszego do ósmego włącznie).

\img{\chapterPath/img/header2.png}{Nagłówek w widoku węzła z informacją o połączeniach}{header2}{.8}

Ostatnią częścią nagłówka jest ikona stanu. Zawiera ona 3 istotne informacje: rodzaj węzła (artykuł lub kategoria), aktualny typ połączeń (wychodzące z węzła lub do niego trafiające) oraz wariant poruszania się. Rodzaj węzła jest ilustrowany poprzez kolor oraz kształt (Rysunek \ref{fig:node_sprites}). O typie połączeń świadczy również kolor, a także skierowanie strzałek w stosunku do symbolu węzła. (Na Rysunku \ref{fig:header2} ikona oznacza artykuł z połączeniami wychodzącymi). Domyślnym wariantem poruszania się jest ręczne przeskakiwanie z węzła na węzeł. Przy uruchomionej zaprogramowanej trasie (czyli automatycznym przechodzeniu po zdefiniowanej ścieżce - więcej o tym w sekcji \ref{sec:history}) na symbolu węzła pojawia się fraza \textit{AUTO} (Rysunek \ref{fig:header3}).

\img{\chapterPath/img/header3.png}{Nagłówek z ikoną stanu sygnalizującą trwające automatyczne przemierzanie trasy}{header3}{.8}

Aplikacja została dodatkowo wyposażona w elementy wspomagające pracę z grafem. Na dolnym ekranie widoczne są półprzezroczyste linie pomocnicze określające płaszczyznę podłoża. Przemieszczają się one z kamerą i mają na celu zapewnienie punktu odniesienia do innych obiektów otaczających użytkownika. Opisana siatka widoczna jest na Rysunku \ref{fig:widok_wezla}.

Ostatnim elementem jest przestrzeń pomocy. Jest to specjalna przestrzeń wydzielona od struktury grafu, do której można wejść w dowolnej chwili korzystania z aplikacji. Użytkownik jest w niej umieszczany również na starcie programu. Zawiera ona podstawowe informacje o programie, a także zestaw infografik zapoznających użytkownika ze sposobami sterowania i możliwościami aplikacji (Rysunek \ref{fig:info_space}).

\img{\chapterPath/img/info_space.png}{Jedna z infografik dostępnych w przestrzeni pomocy}{info_space}{.8}