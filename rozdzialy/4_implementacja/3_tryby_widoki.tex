\section{Tryby poruszania się i widoki węzła}
\sectionauthor{Mikołaj Mirko}
\label{sec:tryby-widoki}
Aplikacja posiada dwa tryby poruszania się: swobodne latanie oraz podróżowanie po węzłach. Pierwszy z nich charakteryzuje się pełną swobodą przemieszczania się. Po wskazaniu kierunku kontrolerem i pochyleniu joysticka do przodu, kamera zaczyna przemieszczać się do przodu, przy odchyleniu joysticka do tyłu w przeciwną stronę. Otaczająca użytkownika przestrzeń jest wizualnie wzorowana na przestrzeni kosmicznej (skybox posiada przygotowaną teksturę imitującą gwiaździste niebo). Rozmieszczone w tej przestrzeni węzły (artykuły i kategorie) można wskazywać i wybierać. W przypadku wskazania jednego z nich wyświetlany jest podgląd jego pierwszych połączeń (jeżeli jakiekolwiek posiada). Rysunek \ref{fig:widok_grafu} prezentuje ten scenariusz na przykładzie artykułu ``Gdańsk''.

\img{\chapterPath/img/widok_grafu.png}{Widok grafu ze wskazanym węzłem i podglądem jego połączeń}{widok_grafu}{.9}

Po wybraniu węzła (naciśnięciu przycisku spustu na kontrolerze) użytkownik umieszczany jest w widoku węzła i wykorzystuje tryb poruszania się po węzłach. W tym momencie kamera umieszczana jest nad wybranym węzłem, a linie połączeń wypełniają się kolorem. Nieaktywne węzły są wygaszane, aby zwiększyć przejrzystość widoku węzła. W każdej chwili można powrócić do widoku grafu i swobodnego latania klikając przycisk oznaczony jako ``Exit'' na kontrolerze. Na Rysunku \ref{fig:widok_wezla} znajduje się przykładowy wygląd wychodzących połączeń z artykułu ``Politechnika Gdańska'' ze wskazanym artykułem ``Technologia'' i podglądem jego połączeń. Przemieszczanie w tym trybie polega głównie na przechodzeniu z węzła na węzeł poprzez dostępne połączenia. Cofanie się po przebytej drodze to inny sposób trawersowania w tym trybie – został on szczegółowo opisany w sekcji \ref{sec:history}.

\img{\chapterPath/img/widok_wezla.png}{Widok węzła z połączeniami i zaznaczonym powiązanym węzłem}{widok_wezla}{.9}

Widok węzła składa się z dwóch części. Pierwsza z nich definiuje wyświetlane połączenia jako wychodzące z wybranego węzła (jest to domyślne rozwiązanie). Drugą częścią są połączenia, które docierają z innych węzłów do aktualnie wybranego. Aby przełączyć się między widokami, należy nacisnąć spust wskazując kontrolerem w aktualnie wybrany element, znajdujący się na dolnym ekranie. Różnice pomiędzy tymi rodzajami połączeń (oraz sposoby ich pozyskiwania) zostały opisane w sekcji \ref{sec:generating-missing-files}.