\section{Zarządzanie pamiecią w aplikacji}
\sectionauthor{Jan Kruczyński}
\label{sec:pamiec}
Ze względu na to, że aplikacja doładowywuje węzły wraz z kolejnymi akcjami użytkownika, gdy przemieszcza się on po grafie lub wskazuje na węzły w przestrzeni, musiał zostać opracowany sposób ich usuwania, by utrzymać płynność działania. Aby to osiągnąć powstał oddzielny kontroler zarządzania pamięcią.

\begin{lstlisting}[caption={Pomocnicze struktury i zmienne kontrolera zarządzania pamięcią}, label=lst:nodePriority]
private List<uint> lowPriorityNodes;%*\label{line:lowPriorityNodes}*)
private List<uint> highPriorityNodes;%*\label{line:highPriorityNodes}*)
public int maxAmountOfNodes;%*\label{line:maxNodes}*)
\end{lstlisting}

Węzły które są załadowane w aplikacji zostały podzielone na dwie kategorie - węzły niskiego priorytetu (\ref{line:lowPriorityNodes}), które są potencjalnymi kandydatami na węzły do usunięcia, oraz wysokiego priorytetu (\ref{line:highPriorityNodes}), których usunąć aktualnie nie można.

Początkowo wszystkie załadowane węzły trafiają do listy niskiego priorytetu. Jeżeli użytkownik dokonuje interakcji z węzłem, to zarówno ten węzeł, jak i wszyscy wyświetlani jego sąsiedzi trafiają do listy wysokiego priorytetu. Jeżeli któryś z tych węzłów znajdował się w liście niskiego priorytetu, jest z niej usuwany i przenoszony na początek listy wysokiego priorytetu.

Funkcjonuje to w taki sposób, aby aplikacja zwalniała w pierwszej kolejności węzły z którymi użytkownik nigdy nie miał interakcji - i dzięki temu nie dostrzegł, że czegoś brakuje. Jednocześnie realizowany jest dodatkowy cel, by ilość wyświetlanych (i przez to przechowywanych w pamięci) węzłów w dowolnej chwili była stała z dokładnością do kilkunastu sztuk.

Kontroler nasłuchuje zdarzenia załadowania węzła przez aplikację. Gdy takie zdarzenie zostanie wykryte, jeżeli liczba wszystkich węzłów, które są wyświetlane w aplikacji, nie przekracza maksymalnej dopuszczonej ilości, nie dzieje się nic. Jeżeli ta liczba zostanie przekroczona, zwalnia kilkanaście węzłów z listy niskiego priorytetu (\ref{line:lowPriorityNodes}).

Gdy lista niskiego priorytetu jest pusta, kontroler przerzuca najstarsze węzły z listy wysokiego priorytetu do listy niskiego priorytetu (są to węzły, z którymi użytkownik prowadził interakcję najdawniej). Gdy użytkownik dokonuje interakcji z węzłem, jest on przerzucany na sam początek listy wysokiego priorytetu.