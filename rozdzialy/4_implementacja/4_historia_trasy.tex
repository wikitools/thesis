\section{Historia przeglądania oraz zaprogramowane trasy}
\sectionauthor{Mateusz Janicki}
\label{sec:history}
Specyfika działania aplikacji, związana z wykonywaniem przez użytkownika akcji nawigacyjnych, oraz środowisko działania skłoniły do zaimplementowania funkcjonalności związanych z historią. Znacząco ułatwią i przyspieszą one poruszanie się po węzłach grafu. Bez możliwości skorzystania z historii użytkownik byłby zmuszony do każdorazowego szukania poprzedniego węzła  w grafie, bądź w odpowiednim widoku wybranego węzła. Co więcej, aby zapewnić powtarzalność podczas prezentacji, a także, aby wyeksponować ciekawe przypadki połączeń pomiędzy węzłami, będzie możliwe zapisanie tras przeglądania. Odtwarzanie tras umożliwia dokładne planowanie prezentacji i eliminuje problemy wynikające z losowości wyświetlania węzłów. 

Do historii zapisywane są tylko akcje kluczowe dla nawigacji: zmiana widoku wyświetlania oraz wybieranie węzłów. Inne działania użytkownika nie powodują znaczących zmian w nawigacji po grafie i ich przechowywanie jest bezcelowe. Po naciśnięciu Przycisku 2 na kontrolerze, uprzednio przytrzymując przycisk spustu, wywoływana jest poprzednia akcja wykonana przez użytkownika. Odpowiednio, aby ponowić cofniętą akcję, należy nacisnąć Przycisk 3. W przypadku braku akcji do ponowienia lub cofnięcia nie jest podejmowane żadne działanie. Wywoływanie i zapisywanie tras dostępne jest z poziomu konsoli operatora opisanej w następnym punkcie.

Akcje użytkownika są przechowywane jako odpowiednie klasy dziedziczące po interfejsie  \codeinline{UserAction}. Posiada on wirtualne funkcje  \codeinline{Execute()} oraz  \codeinline{UnExecute()}, służące odpowiednio ponawianiu i cofaniu akcji. Każda akcja posiada także swojego własnego delegata zależnego od typu akcji. 

\begin{lstlisting}[caption={Interfejs UserAction}, label=lst:IUserAction]
public interface UserAction {
		void Execute();
		void UnExecute();
}
\end{lstlisting}


Bezpośrednie operacje na historii wykonywane są przez  \codeinline{HistoryService}. Przechowuje on akcje użytkownika we dwu stosach: pierwszy służy zapisywaniu akcji do cofnięcia, drugi akcji do ponowienia. W momencie wciśnięcia przycisku 2 wyzwalane jest zdarzenie wywołujące funkcję  \codeinline{UndoAction()}, która, poza wykonaniem kolejnego zdarzenia wywołującego funkcję akcji odpowiedzialnej za cofanie, zdejmuje tę akcję ze stosu akcji do cofania i wkłada na stos akcji do ponawiania.

Wszystkie delegowane funkcje służące historii oraz trasom są przypisywane w kontrolerze  \codeinline{HistoryController}. Warunkiem jest, aby aplikacja była serwerem.

Zaznaczanie węzłów z historii oraz zmiana widoku odbywa się w identyczny sposób, jak zwykłe wywoływanie tych akcji przy użyciu kontrolera, jednak nie są one ponowne zapisywane do historii, jak przy zwykłym wywołaniu.

Analogicznie do modułu historii zaprojektowane zostało odtwarzanie tras. Pliki zawierające trasy posiadają rozszerzenie \textit{.wgroute}. Struktura pliku jest stosunkowo prosta - pierwsza cyfra służy rozróżnieniu akcji użytkownika: 0 oznacza zmianę widoku, a 1 wybranie węzła. W przypadku zmiany widoku po średniku można znaleźć cyfrę 0 lub 1. 0 odpowiada zmianie widoku na widok dzieci, a 1 zmianie widoku na widok rodziców. Jeśli akcją było wybranie węzła, po średniku znajduje się ID artykułu do wybrania. 


\img{\chapterPath/img/wgroute.png}{Fragment przykładowego pliku .wgroute zawierającego trasę}{wg_route}{0.35}

Po wybraniu trasy rozpoczyna się proces jej wczytywania. Są za to odpowiedzialne dwie klasy:  \codeinline{RoutesReader} oraz  \codeinline{RoutesLoader}. Pierwsza ma za zadanie pobrać dostępne trasy, ich nazwy i długość oraz wczytywać kolejne linijki plików. Druga przekształca wczytane dane linijka po linijce na odpowiednie akcje i dopisuje do stosu.

W momencie włączenia trasy inicjowane jest wczytanie trasy z plików, a po tej operacji tworzony jest współprogram(ang. coroutine), który ma za zadanie wykonywanie kolejnych akcji trasy co daną liczbę sekund. Odtwarzanie trasy sygnalizowane jest przez ikonę \textit{AUTO} w nagłówku. W każdym momencie możliwe jest wyłączenie odtwarzania trasy za pomocą Przycisku 4, powodującego zatrzymanie współprogramu. W przypadku włączenia innej trasy w trakcie odtwarzania, współprogram poprzedniej trasy jest także wyłączany, a następnie wczytywana i inicjalizowana jest nowa trasa. 
