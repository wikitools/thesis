\section{Historia przeglądania oraz zaprogramowane trasy}
Opis implementacji oraz działania.

Specyfika działania aplikacji, związana z wykonywaniem przez użytkownika akcji nawigacyjnych, oraz środowisko działania skłoniły do zaimplementowania funkcjonalności związanych z historią. Znacząco ułatwi i przyspieszy to poruszanie się po węzłach grafu. Bez możliwości skorzystania z historii użytkownik byłby zmuszony do każdorazowego szukania poprzedniego węzła  w grafie bądź w odpowiednim widoku wybranego węzła. Co więcej, aby zapewnić powtarzalność podczas prezentacji, a także aby wyeksponować ciekawe przypadki połączeń pomiędzy węzłami, będzie możliwe zapisanie tras przeglądania. Odtwarzanie tras umożliwia dokładne planowanie prezentacji i eliminuje problemy wynikające z losowości wyświetlania węzłów. 

Do historii zapisywane są tylko akcje kluczowe dla nawigacji - zmiana widoku wyświetlania oraz wybieranie węzłów. Inne działania użytkownika nie powodują znaczących zmian w nawigacji po grafie i ich przechowywanie jest bezcelowe. Po naciśnięciu Przycisku 2 na kontrolerze wywoływana jest poprzednia akcja wykonana przez użytkownika. Odpowiednio, aby ponowić cofniętą akcję, należy nacisnąć Przycisk 3. W przypadku braku akcji do ponowienia lub cofnięcia nie jest podejmowane żadne działanie. Wywoływanie i zapisywanie tras dostępne jest z poziomu konsoli operatora opisanej w następnym punkcie. (Much to do)

Akcje użytkownika są przechowywane jako odpowiednie klasy dziedziczące po klasie UserAction. Posiada ona wirtualne funkcje Execute() oraz UnExecute() służące odpowiednio ponawianiu i cofaniu akcji. Bezpośrednie operacje na historii wykonywane są przez HistoryService. Przechowuje on akcje użytkownika we dwu stosach: pierwszy służy zapisywaniu akcji do cofnięcia, drugi akcji do ponowienia. W momencie wciśnięcia przycisku 2 wywoływana jest funkcja …, która poza wykonaniem zdarzenia wywołującego cofnięcie akcji, zdejmuje tę akcję ze stosu akcji do cofania i wkłada na stos akcji do ponawiania.

Analogicznie do modułu historii zaprojektowane zostało odtwarzanie tras. Pliki zawierające trasy posiadają rozszerzenie .wgroute. Struktura pliku jest prosta - pierwsza cyfra służy rozróżnieniu akcji użytkownika. Jeśli jest to akcja wybrania węzła, to po średniku wpisywane jest ID węzła, który został wybrany. 



