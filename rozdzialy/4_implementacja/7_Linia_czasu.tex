\section{Linia czasu}
\sectionauthor{Mateusz Janicki}
Specyfikacja wymagań projektu inżynierskiego zawierała funkcjonalność dotyczącą linii czasu. Wybranie daty miało spowodować wyświetlanie tylko takich węzłów, których data utworzenia była starsza niż wybrana data. Pomogłoby to w przybliżonej wizualizacji w jak szybki sposób rozrastała się konkretna Wikipedia. Szereg komplikacji spowodował jednak odrzucenie przez nas tej funkcjonalności.

Sterowanie linią czasu miało odbywać się za pomocą drugiego kontrolera. Po wciśnięciu przycisku miał pokazywać się specjalny interfejs ukazujący aktualnie wybraną datę. Można w niej było wybrać, przesuwając joystick w lewo i prawo, miesiąc lub rok, który następnie dałoby się zmieniać przesuwając joystick do góry lub w dół. Zatwierdzenie daty przyciskiem spustu wywoływałoby zmiany w grafie. 

Funkcjonalność nie została zaimplementowana głównie z powodu braku prostej informacji o stworzeniu artykułu w głównie wykorzystywanym przez nas źródle danych, czyli dumpach WIkipedii. Istnieją dumpy zawierającą kompletną historię edycji artykułów, jednak wielkość tych plików jest nieporównywalnie większa w porównaniu ze zwykłym wykazem stron czy nawet połączeniami pomiędzy stronami. Przetworzenie tych plików tylko po to, aby wydobyć z nich datę pierwszej publikacji strony jest niewymierna do korzyści wynikających z tej funkcjonalności. 

Innym sposobem wydobycia danych mogłoby być pozyskiwanie tych informacji przy użyciu Wikipedia API. Pomysł ten jednak nie jest wykonalny z powodu ograniczeń, które posiada API. Aby zdobyć informacje dotyczące każdego artykułu, musielibyśmy wywołać zapytania, których liczba znacznie przewyższa maksymalną dopuszczalną liczbę zapytań dla jednego użytkownika. Przekroczenie tej liczby powoduje zablokowanie adresu IP, z którego pochodziły zapytania i utrudnia dalsze pozyskiwanie danych.

Co więcej, nasza idea linii czasu nie byłaby dokładnym odwzorowaniem stanu Wikipedii w wybranym przez użytkownika. Artykuły i kategorie WIkipedii są bardzo często zmieniane, co implikuje możliwość zmiany połączeń pomiędzy artykułami. Ponadto artykuły i kategorie mogą być także usuwane. Przechowywanie całej historii stron lub stworzenie plików zawierających dokładną historię wszystkich węzłów, jakie kiedykolwiek pojawiły się na Wikipedii, jest niemożliwe na zwykłych komputerach osobistych z powodu zbyt małej wymaganej przestrzeni na dysku. Nawet jeśli udałoby się je zapisać, przetwarzanie i wczytywanie takich plików zajęłoby bardzo dużo czasu i spowodowałoby niską responsywność aplikacji. Nasza wersja linii czasu mogłaby być myląca dla nowych użytkowników, którzy mogliby pomyśleć, że przedstawiana jest im dokładna wersja Wikipedii w danym czasie.

Wszystkie napotkane trudności sprawiły, że ostatecznie zrezygnowaliśmy z implementacji tej funkcjonalności. Dodatkowo dzięki temu możliwa jest intuicyjna obsługa aplikacji za pomocą tylko jednego kontrolera, co znacznie ułatwia użytkownikowi sterowanie w jaskini. Zmiany w sterowaniu zostały ujęte na rysunku 4.6.

\img{\chapterPath/img/nowy_schemat_kontrolera.png}{Aktualny schemat sterowania kontrolerem}{flystick_new_controls}{0.8}

