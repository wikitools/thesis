\section{Wizualna reprezentacja grafu}
\sectionauthor{Stanisław Góra}
\label{sec:graf-reprezentacja}

\newcommand\mapitem[3]{
	\item \textbf{#1 $\rightarrow$ #2}

	#3
}

% Przechowywanie grafu w pamięci
\noindent
Graf jest przechowywany w pamięci aplikacji jako zbiór map:
\begin{enumerate}[label=\textbullet]
	\mapitem{Identyfikator węzła}{Model węzła}{Opisana już mapa warstwy modelu}
	\mapitem{Model węzła}{Obiekt wizualny węzła}{Przechowuje widoczne aktualnie w aplikacji węzły jako wartości przypisane do modelu reprezentowanego węzła. Obiekt wizualny przechowuje numer ID węzła, co zamyka pętlę powiązań i w efekcie pozwala na uzyskanie informacji o węźle (z jego modelu), mając na wejściu jego obiekt wizualny na podstawie dwóch map.}
	\mapitem{Model połączenia}{Obiekt wizualny połączenia}{Mapa analogiczna do poprzedniej, przechowująca informacje o połączeniach między węzłami. Podobnie obiekt wizualny połączenia przechowuje dwa identyfikatory węzłów które łączy.}
	
	\mapitem{Model połączenia}{Obiekt wizualny reprezentacji węzła końcowego połączenia}{Mapa przechowująca pomocnicze reprezentacje węzłów końcowych połączenia opisane szczegółowo w sekcji \ref{sec:tryby-widoki}.}
\end{enumerate}

% GO Node - Billboard Shader
\subsection{Reprezentacja węzła} Z uwagi na dużą ilość jednocześnie załadowanych węzłów (kilka do kilkunastu tysięcy na raz) konieczne okazały się pewne optymalizacje, poprawiające szybkość działania aplikacji. 

Obiekty węzłów nie są w większości przypadków tworzone dynamicznie. Zamiast tego przy starcie aplikacji tworzona jest pula nieużywanych obiektów. W momencie ładowania węzła nowy obiekt wizualny tworzony jest tylko w przypadku, kiedy pula została wyczerpana. Przy usuwaniu węzła jego reprezentacja nie jest niszczona, tylko zwracana do puli w celu przyszłego wykorzystania.

W celu odciążenia karty graficznej zrezygnowaliśmy z wyświetlania każdego węzła jako trójwymiarowego modelu (jak na przykład kuli). Reprezentacją węzła jest prosta grafika rastrowa (rysunek \ref{fig:node_sprites}). Dzięki temu siatka przetwarzana przez komputer jest dużo prostsza, ponieważ zawiera tylko cztery wierzchołki (dwa trójkąty) - jest to kwadratowa płaszczyzna z nałożoną teksturą. Dzięki temu aplikacja jest w stanie wyświetlić więcej węzłów na raz bez utraty płynności działania.

\img{\chapterPath/img/node_sprites.png}{Grafiki węzłów (od lewej): artykuł aktywny i wskazany, kategoria aktywna i wskazana}{node_sprites}{0.5}

Takie rozwiązanie powoduje jednak pewną komplikację. Płaszczyzna z grafiką, w przeciwieństwie do przestrzennego modelu, jest dobrze widoczna tylko kiedy kamera jest ustawiona bezpośrednio przed nią i zwrócona w jej stronę. Pojawia się więc potrzeba obracania obiektu węzła tak, aby był on zawsze zwrócony przodem do użytkownika zwiedzającego graf. Najprostszym rozwiązaniem byłoby dodanie krótkiego skryptu do każdego obiektu, który przy każdym odświeżeniu ekranu obracałby reprezentację węzła w stronę kamery. Nie jest to jednak dobre rozwiązanie z powodu narzutu czasowego. W każdej klatce ta sama operacja byłaby wykonywana potencjalnie kilkanaście tysięcy razy, co mogłoby mieć widoczny efekt na szybkości działania aplikacji.

Znacznie lepszym rozwiązaniem jest użycie shadera. W przeciwieństwie do standardowych skryptów aplikacji jego działanie jest zrównoleglone przez kartę graficzną dla potencjalnie każdego przetwarzanego wierzchołka. W środowisku Unity do pisania Shaderów wykorzystuje się język \codeinline{ShaderLab} oraz \codeinline{HLSL/Cg} \cite{UnityShaders}. Podczas pierwszych testów tego rozwiązania pojawił się jednak kolejny problem. Powszechnie stosowane są tak zwane ``Billboard Shadery'', które służą dokładnie do interesującego nas celu, jednak w tym szczególnym przypadku aplikacji budowanej na środowisko LZWP są one niewystarczające. 

\paragraph{Problem} Każda ściana jaskini rzeczywistości wirtualnej jest kontrolowana przez osobny komputer z własną instancją aplikacji, posiadającą swoją parę kamer (dla każdego oka) skierowaną w kierunku danej ściany w przestrzeni wizualizacji. Ze względu na wymagania stereoskopii oraz spójności obrazów na każdej ze ścian, kamery te nie znajdują się w dokładnie tym samym punkcie w przestrzeni symulacji. Są od siebie lekko oddalone (rozstaw oczu dla każdej ściany oraz obrót głowy dla każdej pary przylegających ścian). Billboard Shader odwraca obiekty węzłów tak, aby były zwrócone przodem do punktu, w którym znajduje się kamera na lokalnej ścianie. Przykład został zilustrowany na rysunku \ref{fig:billboard-schema}. Powoduje to, że jeśli dany węzeł znajdzie się w pozycji wyświetlanej na przecięciu ekranów (na przykład połowa węzła na jednej a połowa na drugiej ścianie jaskini), na każdym z odpowiadających za ściany komputerów zostanie obrócony w nieco innym kierunku. Różnica kątów obrotu \(\alpha\) będzie tu odpowiadała odległości kątowej pomiędzy położeniem kamer obu ścian z perspektywy węzła. W efekcie części obiektu węzła mogą nie być spójne z punktu widzenia obserwatora znajdującego się w jaskini.

\img{\chapterPath/img/billboard-schema.png}{Poglądowy schemat ilustrujący problem, widok z góry\\ (prawidłowa skala nie została przedstawiona)}{billboard-schema}{.4}

\paragraph{Rozwiązanie} Aby rozwiązać ten problem, należało wykonywać obrót grafik węzłów względem jednego punktu znajdującego się w dokładnie tej samej pozycji dla wszystkich ścian (użyta została centralna pozycja głównych okularów śledzonych w jaskini). To przekształcenie wymagało ręcznego przepisania shadera z powodu braku gotowego i dostępnego rozwiązania. Problem sprowadził się do stworzenia macierzy \(R\) (wzór \ref{eq:macierz_wzor} oraz \ref{eq:macierz}) przekształcającej wektor trójwymiarowy \(a\) w \(b\) - w tym przypadku domyślny zwrot grafiki węzła w kierunek wskazujący na pozycję nowego wybranego obiektu wspólnego. Implementacja jest wzorowana na wpisie z forum matematycznego \cite{BillboardShaderFormula}.

\newcommand\dotProd{a \cdot b}
\newcommand\crossProd[1]{(\overrightarrow{a \times b})_#1}
\begin{equation}
	\label{eq:macierz_wzor}
	R = I + [v]_{\times} + [v]_{\times}^2\frac{1}{1 + c}
\end{equation}
\begin{equation}
	\label{eq:macierz}
	[v]_{\times} \stackrel{\rm def}{=} 
	\begin{bmatrix}
		0 & -v_z & v_y\\
		v_z & 0 & -v_x\\
		-v_y & v_x & 0
	\end{bmatrix},
\end{equation}
gdzie:
\begin{itemize}
	\item \(v = a \times b\)
	\item \(c = a \cdot b\)
\end{itemize}

% GO Connection - B-Spline
\subsection{Reprezentacja połączenia}
Połączenia są w aplikacji reprezentowane jako krzywe (rysunek \ref{fig:node-connections}). Ich kształt został najpierw zawinięty w górę tak, aby przechodzić przez stworzone dla każdego połączenia węzły pomocnicze opisane w sekcji \ref{sec:tryby-widoki}, a następnie zmierza łukiem w kierunku faktycznego węzła końcowego. Na początku do tworzenia tras połączeń używane były krzywe Béziera, jednak obecnie, z powodu ich skomplikowanego kształtu, zastąpione zostały krzywymi B-Spline zaimplementowanymi według algorytmu De Boora.

\img{\chapterPath/img/connections.png}{Wygląd węzłów i połączeń w aplikacji}{node-connections}{.9}
