\section{Wizualna reprezentacja węzłów i połączeń}
\label{sec:graf-reprezentacja}
\newcommand\mapitem[3]{
	\item \textbf{#1 $\,\to\,$ #2}

	#3
}

% Przechowywanie grafu w pamięci
\noindent
Graf jest przechowywany w pamięci aplikacji jako zbiór map:
\begin{enumerate}[label=\textbullet]
	\mapitem{Identyfikator węzła}{Model węzła}{Opisana już mapa warstwy modelu}
	\mapitem{Model węzła}{Obiekt wizualny węzła}{Przechowuje widoczne aktualnie w aplikacji węzły jako wartości przypisane do modelu reprezentowanego węzła. Obiekt wizualny przechowuje numer ID węzła co zamyka pętlę powiązań i w efekcie pozwala na uzyskanie informacji o węźle (z jego modelu) mając na wejściu jego obiekt wizualny na podstawie dwóch map.}
	\mapitem{Model połączenia}{Obiekt wizualny połączenia}{Mapa analogiczna do poprzedniej, przechowująca informacje o połączeniach między węzłami. Podobnie obiekt wizualny połączenia przechowuje dwa identyfikatory węzłów które łączy.}
	
	\mapitem{Model połączenia}{Obiekt wizualny reprezentacji węzła końcowego połączenia}{Mapa przechowująca pomocnicze reprezentacje węzłów końcowych połączenia opisane szczegółowo w \ref{sec:tryby-widoki}.}
\end{enumerate}

% GO Node - Billboard Shader
% GO Connection - B-Spline

