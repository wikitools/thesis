\section{Wizualna reprezentacja grafu}
\sectionauthor{Stanisław Góra}
\label{sec:graf-reprezentacja}

\newcommand\mapitem[3]{
	\item \textbf{#1 $\to$ #2}

	#3
}

% Przechowywanie grafu w pamięci
\noindent
Graf jest przechowywany w pamięci aplikacji jako zbiór map:
\begin{enumerate}[label=\textbullet]
	\mapitem{Identyfikator węzła}{Model węzła}{Opisana już mapa warstwy modelu}
	\mapitem{Model węzła}{Obiekt wizualny węzła}{Przechowuje widoczne aktualnie w aplikacji węzły jako wartości przypisane do modelu reprezentowanego węzła. Obiekt wizualny przechowuje numer ID węzła co zamyka pętlę powiązań i w efekcie pozwala na uzyskanie informacji o węźle (z jego modelu) mając na wejściu jego obiekt wizualny na podstawie dwóch map.}
	\mapitem{Model połączenia}{Obiekt wizualny połączenia}{Mapa analogiczna do poprzedniej, przechowująca informacje o połączeniach między węzłami. Podobnie obiekt wizualny połączenia przechowuje dwa identyfikatory węzłów które łączy.}
	
	\mapitem{Model połączenia}{Obiekt wizualny reprezentacji węzła końcowego połączenia}{Mapa przechowująca pomocnicze reprezentacje węzłów końcowych połączenia opisane szczegółowo w sekcji \ref{sec:tryby-widoki}.}
\end{enumerate}

% GO Node - Billboard Shader
\subsection{Reprezentacja węzła} Z uwagi na dużą ilość jednocześnie załadowanych węzłów (kilka do kilkunastu tysięcy na raz) konieczne okazały się pewne optymalizacje poprawiające szybkość działania aplikacji. 

Obiekty węzłów nie są w większości przypadków tworzone dynamicznie. Zamiast tego przy starcie aplikacji tworzona jest pula nieużywanych obiektów. W momencie ładowania węzła nowy obiekt wizualny tworzony jest tylko w przypadku kiedy pula została wyczerpana. Przy usuwaniu węzła jego reprezentacja nie jest niszczona tylko zwracana do puli w celu przyszłego wykorzystania.

W celu odciążenia karty graficznej zrezygnowaliśmy z wyświetlania każdego węzła jako trójwymiarowego modelu (jak na przykład kuli). Reprezentacją węzła jest prosta grafika rastrowa (Rysunek \ref{fig:node_sprites}). Jest ona dużo szybsza do przetworzenia ponieważ zawiera tylko cztery wierzchołki - jest to kwadratowa płaszczyzna z nałożoną teksturą. Dzięki temu aplikacja jest w stanie wyświetlić więcej węzłów na raz bez utraty płynności działania.

\img{\chapterPath/img/node_sprites.png}{Grafiki węzłów (od lewej): artykuł aktywny i wskazany, kategoria aktywna i wskazana}{node_sprites}{0.6}

Takie rozwiązanie powoduje jednak pewną komplikację. Płaszczyzna z grafiką, w przeciwieństwie do przestrzennego modelu, jest dobrze widoczna tylko kiedy kamera jest ustawiona bezpośrednio przed nią. Pojawia się więc potrzeba obracania obiektu węzła tak aby był on zawsze zwrócony przodem do użytkownika zwiedzającego graf. Najprostszym rozwiązaniem byłoby dodanie krótkiego skryptu do każdego obiektu który przy każdym odświeżeniu ekranu obracałby reprezentację węzła w stronę kamery. Nie jest to jednak dobre rozwiązanie z powodu narzutu czasowego. W każdej klatce ta sama operacja byłaby wykonywana potencjalnie kilkanaście tysięcy razy co mogłoby mieć widoczny efekt na szybkości działania.

Poprawnym rozwiązaniem jest użycie shadera. W przeciwieństwie do standardowych skryptów aplikacji jego działanie jest zrównoleglone przez kartę graficzną dla potencjalnie każdego przetwarzanego wierzchołka. Tu jednak natrafiliśmy na kolejny problem. Powszechnie stosowane są tak zwane ``Billboard Shadery'' które służą dokładnie do interesującego nas celu, jednak w szczególnym przypadku aplikacji budowanej na środowisko LZWP są niewystarczające. Każda ściana jaskini jest kontrolowana przez osobny komputer z osobną instancją aplikacji posiadającą osobną kamerę skierowaną w kierunku danej ściany. Billboard Shader odwraca obiekty węzłów w kierunku kamery na lokalnej ścianie co powoduje że jeśli dany obiekt znajdzie się w pozycji wyświetlanej na przecięciu ekranów (połowa węzła na jednej a połowa na drugiej ścianie jaskini), na każdym z odpowiadających za ściany komputerów zostanie obrócony w lekko innym kierunku. W efekcie części obiektu węzła mogą nie być spójne.

Aby rozwiązać ten problem należało wybrać obiekt wspólny dla wszystkich ścian a następnie obracać grafiki węzłów w względem niego zamiast kamer. To wymagało ręcznego przepisania shadera z powodu braku gotowego i dostępnego rozwiązania. Problem sprowadził się do stworzenia macierzy \(R\) przekształcającej wektor trójwymiarowy \(a\) w \(b\) - w tym przypadku domyślny zwrot grafiki węzła w kierunek wybranego obiektu wspólnego. Implementacja jest wzorowana na wpisie z forum matematycznego \cite{BillboardShaderFormula}.

\newcommand\dotProd{a \cdot b}
\newcommand\crossProd[1]{(\overrightarrow{a \times b})_#1}
\begin{equation}
	R = I + [v]_{\times} + [v]_{\times}^2\frac{1}{1 + c}
\end{equation}
\begin{equation}
	[v]_{\times} \stackrel{\rm def}{=} 
	\begin{bmatrix}
		0 & -v_z & v_y\\
		v_z & 0 & -v_x\\
		-v_y & v_x & 0
	\end{bmatrix}
\end{equation}
gdzie:
\begin{itemize}
	\item \(v = a \times b\)
	\item \(c = a \cdot b\)
\end{itemize}

% GO Connection - B-Spline
\subsection{Reprezentacja połączenia}
Połączenia są w aplikacji reprezentowane jako krzywe (Rysunek \ref{fig:node-connections}). Ich kształt został najpierw zawinięty w górę tak aby przechodzić przez stworzone dla każdego połączenia węzły pomocnicze opisane w sekcji \ref{sec:tryby-widoki} a następnie zmierza łukiem w kierunku faktycznego węzła końcowego. Na początku do tworzenia tras połączeń używane były krzywe Béziera jednak obecnie z powodu ich skomplikowanego kształtu zastąpione zostały krzywymi B-Spline zaimplementowanymi według algorytmu De Boor'a.

\img{\chapterPath/img/connections.png}{Wygląd węzłów i połączeń w aplikacji}{node-connections}{.9}
