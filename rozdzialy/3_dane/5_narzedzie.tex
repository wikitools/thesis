\section{Narzędzie WikiGraph Parser}
\sectionauthor{Mikołaj Mirko}
\label{sec:parser-tool}
Proces pozyskiwania danych opisany w podrozdziałach \ref{sec:data-source} - \ref{sec:data-files} został zautomatyzowany i zaimplementowany w postaci dodatkowego, pomocniczego programu. Aplikacja WikiGraph Parser ma na celu usprawnienie pracy z pozyskiwanymi danymi oraz zmniejszenie ryzyka wystąpienia nieprawidłowości. Aplikacja została napisana w języku C\# (z platformą docelową .NET Framework 4.6.1) i posiada interfejs graficzny stworzony za pomocą frameworka UI WPF.

Interfejs programu został zaprojektowany z myślą o 10 heurystykach Nielsena \cite{Heuristics}. Zastosowano estetyczny i minimalistyczny wygląd wykorzystujący typowe standardy aplikacji okienkowych. W wielu miejscach użytkownik jest informowany o stanie programu za pomocą etykiet oraz symboli graficznych. Walidacja pozwala uniknąć przewidywalnych błędów, a w razie wystąpienia nieoczekiwanych problemów pojawiają się opisowe okna dialogowe. Pomoc zawarta jest w postaci tytułów sekcji i ich podpowiedzi.

\img{\chapterPath/img/parser1.png}{Ekran konfiguracji parametrów WikiGraph Parser-a}{parser1}{0.7}

Użytkownik obsługujący WikiGraph Parser ma możliwość wyboru interesującego go źródła informacji oraz określenie ścieżki aplikacji, wykorzystującej stworzone zasoby (Rysunek \ref{fig:parser1}). Ograniczając go do wyboru tylko tych dwóch parametrów, tworzony jest prosty interfejs, który zachowuje jednocześnie swobodę i kontrolę użytkowania. Giles Colborne w swojej książce ``Prostota i użyteczność'' \cite{ProstotaUzytecznosc} opisuje zasadę zachowania złożoności, według której zabiegi usuwania i ukrywania funkcjonalności przed użytkownikiem sprowadzają interfejs do jego minimalnego poziomu złożoności.

\begin{center}
	\hyphenblockcquote{polish}{ProstotaUzytecznosc}{
        Cała sztuka projektowania prostych rozwiązań polega na przenoszeniu złożoności w odpowiednie miejsce, tak aby korzystanie z samego narzędzia było łatwe.
	}
\end{center}

\img{\chapterPath/img/parser2.png}{Ekran statusu postępu przetwarzania danych}{parser2}{0.7}

Przetwarzanie danych jest rozpoczynane po naciśnięciu przycisku \textit{Start}. Na ekranie z informacją o aktualnym postępie (Rysunek \ref{fig:parser2}) można wyróżnić pięć kroków. Są to:

\begin{enumerate}[label=\textbullet]
    \item \textit{Downloading dumps}

    Na podstawie wybranego zrzutu bazy pobierany jest odpowiedni zestaw plików (szczegółowo opisany w podrozdziale \ref{sec:data-download}). Do tego zadania wykorzystana jest klasa \codeinline{WebClient}.

    \item \textit{Decompressing dumps}

    Dekompresja pobranych plików odbywa się za pomocą klasy \codeinline{GzipStream}. Po tym kroku pliki \codeinline{.sql} gotowe są do przetwarzania.

    \item \textit{Reading data from dumps}

    Podczas tego kroku tworzone są pliki z rozszerzeniem \codeinline{.map}, zawierające tylko te dane, które są wykorzystywane do dalszego przetwarzania. Na tym etapie odbywa się również sortowanie danych – użyty jest pakiet \codeinline{Sortiously}\footnote{Pakiet użytkownika fredgdaley2 dostępny jest w serwisie GitHub: https://github.com/fredgdaley2/Sortiously}.

    \item \textit{Mapping graph data}

    Ten krok przetwarza przerobione i uporządkowane dane uzupełniając je o brakujące odwrotne połączenia. Problem ten został opisany w sekcji \ref{sec:generating-missing-files}.

    \item \textit{Generating WikiGraph files}	

    Na koniec tworzone są pliki grafu o strukturze binarnej, czytane przez aplikację główną (podrozdział \ref{sec:data-files}). Wykorzystane są m.in. klasy takie jak \codeinline{BinaryWriter} oraz \codeinline{BinaryConverter}. Po zakończeniu poprzednich etapów, niepotrzebne, tymczasowe pliki zostają usunięte z dysku.

\end{enumerate}