\section{Tworzenie struktury grafu}
\sectionauthor{Jan Kruczyński}
\label{sec:data-files}

Celem tego etapu jest wytworzenie plików na których operuje aplikacja. Podczas projektowania ich struktury wzięliśmy pod uwagę, że muszą one:
Zawierać w sobie pełną informację na temat struktury skierowanego grafu połączeń pomiędzy kolejnymi węzłami.
Rozróżniać czy dany węzeł reprezentuje artykuł czy kategorię.
Przechowywać tytuły z wikipedii każdego węzła.
Przechowywać ID strony na wikipedii.
Umożliwiać szybkie odnajdywanie interesujących nas danych o konkretnym węźle bez przeszukiwania wszystkich plików za każdym razem, gdy potrzebujemy wydobyć jakąś informację.
Przechowywać dane w skompresowanej formie, bez zbędnych bajtów i redundantnej informacji.

Struktura owych plików powinna dać nam możliwość funkcjonowania aplikacji, bez trzymania wszystkich danych w pamięci tymczasowej. Odczytywanie danych z plików powinno być możliwie najszybsze.

\subsection{Generowanie brakujących danych}

Dane linków zawierają wyłącznie połączenia typu “z węzła - do innego węzła”, ale nie mają połączeń odwrotnych “do węzła z innych węzłów”. Aby aplikacja była w stanie zaprezentować pełny skierowany graf połączeń potrzebne jest odwzorowanie odwrotne.

Do finalnego poprawnego generowania plików potrzebne są dodatkowe pliki:
Artykuły które prowadzą do konkretnego artykułu (odwrotność pagelinks.map)
Kategorie do których należy dany artykuł (odwrotność categorylinksfrompage.map)
Kategorie do których należy dana kategoria (odwrotność categorylinksfromcategory.map)



\subsection{Opis poszczególnych plików}
Aby osiągnąć wymagania które sobie postawiliśmy, opracowaliśmy strukturę rozbitą na 4 pliki. Wszystkie posiadają tą samą nazwę - jest to nazwa wersji wikipedii którą opisują (np. simple-wiki lub pl-wiki), a różnią się rozszerzeniami - gdyż każdy plik posiada inną strukturę.

3.4.2.1 Plik mapy .wgm
Jest to plik instruujący aplikację na którym miejscu w innych plikach odnajdzie interesujące nas dane.
Każdy węzeł zawiera swój wpis w pliku .map zajmując dokładnie 12 bajtów o poniższej strukturze. Offset informuje nas, od którego bajtu w danym pliku możemy odczytywać informację o danym węźle.

4 BYTES
4 BYTES
4 BYTES
Offset w pliku graph.wg
Offset w pliku titles.wg
ID Wikipedii danej strony

W tym miejscu następuje swoiste przekonwertowanie ID Wikipedii danej strony na “nasze” ID - czyli kolejność danego węzła w pliku .map. W aplikacji oraz w pozostałych plikach, gdy odnosimy się do jakiegoś węzła nie używamy jego faktycznego ID wikipedii, tylko właśnie tą kolejność - licząc od zera. Znając ją, możemy pomnożyć ją przez rozmiar każdego wpisu (12) i otrzymujemy Offset w pliku .map.
Podczas konstrukcji plików, informacje o danym węźle są jednocześnie umieszczane we wszystkich plikach. Dzięki temu nie ma potrzeby przechowywać danej informującej ile bajtów należy odczytać. Odczytu należy dokonać aż do pozycji Offset który znajduje się w następnym węźle z pliku .wgm - czyli 12 bajtów dalej.

3.4.2.2 Plik struktury grafu .wgg

Jest to plik zawierający dane połączeń skierowanego grafu. Rozróżniane są połączenia: (#1) od których węzłów możemy dojść do aktualnego węzła oraz (#2) do których węzłów prowadzi nasz węzeł.

2 BYTES
3 BYTES * N
3 BYTES * X
Ilość połączeń typu (#1) (N)
a#1 b#2 c#3 … z#N
A#1 B#2 C#3 … Z#X
A, B, C oraz a, b, c to nie są ID artykułów wikipedii, lecz informacje, który z kolei artykuł z pliku .wgm mamy na myśli. Jako że ilość węzłów nigdy nie przekracza 224, 3 bajty wystarczają na przekazanie tej informacji.

3.4.2.3 Plik informacyjny .wgi

Zawiera wyłącznie jedną liczbę typu int - jest to ilość artykułów które znajdują się w plikach. Jest to jedyne rozróżnienie dla aplikacji który węzeł traktować jako artykuł, a który jako kategorię - w pozostałych plikach nie ma pomiędzy nimi rozróżnienia. Do plików najpierw zapisujemy wszystkie artykuły, a dopiero potem kategorie, dzięki czemu od pewnego momentu aplikacja oznacza węzły do których pobierana jest informacja jako kategorię. 

3.4.2.4 Plik tytułów .wgt

Zawiera w sobie zapisane w formacie UTF-8 tytuły wszystkich artykułów i kategorii.

3.4.3 Metoda generowania plików

Program generujący pliki .wgX został napisany w języku C#
