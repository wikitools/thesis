\begin{chapter}{Przygotowywanie danych wejściowych}
	\newcommand{\chapterPath}{rozdzialy/3_dane}
	\label{ch:dane}

	Bardzo ważną częścią naszego projektu są dane. Zanim omówiona zostanie wizualizacja i integracja ze środowiskiem jaskini, warto określić podstawowe źródło danych, a także skupić się na procesie przygotowywania informacji do głównej aplikacji. Kolejny rozdział, opisujący jej implementację, będzie wykorzystywał stworzone na tym etapie dane wejściowe.

	\section{Źródło danych}
\sectionauthor{Mikołaj Mirko}
\label{sec:data-source}
Internetowa encyklopedia Wikipedia to jeden z projektów organizacji Wikimedia Foundation. Projekt ten ma na celu gromadzenie, porządkowanie i weryfikowanie otwartych danych tworzonych przez społeczność wolontariuszy (często nazywanych „Wikipedystami”). Według Podstawowego Rankingu Międzyjęzykowego \cite{Wiki:PodstawowyRanking} liczba artykułów w angielskiej wersji językowej na dzień 1 listopada 2019 r. wynosi prawie 6 milionów, a co miesiąc przybywa blisko 20 tysięcy nowych pojęć. Przechowywanie takiej ilości danych (uwzględniając całą treść i media artykułu, znajdujące się w nim powiązania wewnętrzne i zewnętrzne oraz historię jego edycji) jest nie lada wyzwaniem.

Wikimedia Foundation prowadzi również inne inicjatywy, takie jak m.in. Wikibooks (zbiór książek i podręczników), Wikinews (dziennik wydarzeń) i Wikiquote (kolekcja rozmaitych cytatów) – każda z nich posiadająca wiele wersji językowych. Wszystkie oferowane przez fundację serwisy opierają się na oprogramowaniu MediaWiki. Odpowiada ono za ogólną strukturę strony typu wiki, oferując jednocześnie wiele dodatkowych mechanizmów ułatwiających pracę z dużą ilością danych. Skonfigurowane zewnętrzne API pozwala na dostęp do danych innym oprogramowaniem, użycie szablonów stron ułatwia oddzielenie warstwy wizualnej od samych danych, a moduł archiwizacji odpowiada za tworzenie kopii zapasowych baz danych.

Ostatnia z przytoczonych funkcjonalności odgrywa dużą rolę w sposobie zdobycia danych do aplikacji. Kolejny projekt fundacji, o którym trzeba wspomnieć, to Meta-Wiki. Odpowiedzialny jest on za koordynację wszystkich pozostałych projektów. Zawiera bogatą dokumentację, historię zmian i aktualizacji oraz raporty aktywności. Udostępnia również publiczne zrzuty baz danych, zawierające część informacji z każdej dostępnej wersji językowej każdego projektu. Wykonywane są one z częstotliwością około 1 raz na 2 tygodnie i używają wspominanego modułu archiwizacji (choć nie są typową kopią zapasową całego serwisu).

Wśród oferowanych danych w zrzutach można znaleźć przede wszystkim informacje o stronach (czyli artykułach, kategoriach, szablonach, przestrzeniach nazw i kilku innych), historii ich edycji oraz wewnętrznych połączeniach między stronami. Oprócz tego dostępne są różnego rodzaju listy, statystyki i metadane. Niektóre z nich dostępne są w formatach takich jak \codeinline{.txt}, \codeinline{.json} i \codeinline{.xml}, ale większość z nich zapisana jest w postaci plików \codeinline{.sql}, zawierających definicję tabeli bazy danych oraz ciąg wpisów z danymi. Wszystkie udostępniane pliki są dodatkowo zarchiwizowane za pomocą programu GZIP.
	\section{Pobieranie i dekompresja}
\sectionauthor{Mikołaj Mirko}
\label{sec:data-download}
Serwis Wikimedia Downloads oprócz skompresowanych plików zrzutów posiada również plik o nazwie \textit{index.json}. Zawiera on spis wszystkich ostatnio wykonanych operacji archiwizacyjnych na bazie danych. Na jego podstawie łatwo uzyskać bezpośrednie adresy URL do interesujących nas plików, a także dodatkowe informacje o ich rozmiarze i dacie stworzenia. Zawiera on również status przetwarzania każdej porcji danych – wszystkie pobierane pliki muszą być zakończone w ramach tego samego zrzutu, inaczej nie będą ze sobą zgodne, co uniemożliwi generowanie plików wejściowych aplikacji. Listing \ref{lst:dump_index} przedstawia przykładowy fragment pliku \textit{index.json}.

\begin{lstlisting}[frame=single,caption={Fragment informacji o ostatnim zrzucie bazy danych polskiej Wikipedii},label=lst:dump_index]
    "plwiki": {
        "jobs": {
          //...
          "pagetable": {
            "files": {
              "plwiki-20191101-page.sql.gz": {
                "size": 112182889,
                "md5": "d59ca88559792b2520f50368b4c3815a",
                "sha1": "49d16053f695cb27134cae278b1269c6e250445a",
                "url": "/plwiki/20191101/plwiki-20191101-page.sql.gz"
              }
            },
            "updated": "2019-11-02 09:03:12",
            "status": "done"
          },
          //...
        },
        "version": "0.8"
      }      
\end{lstlisting}

Na potrzeby naszej aplikacji potrzebujemy danych z następujących trzech plików:

\begin{enumerate}[label=\textbullet]
    \item \textit{page.sql.gz} – posiada identyfikatory stron artykułów i kategorii, ich przestrzenie nazw oraz tytuły (reszta informacji nie jest wykorzystywana),
    \item \textit{pagelinks.sql.gz} – zawiera spis wewnętrznych połączeń między artykułami - są to odnośniki znajdujące się w treści artykułu, prowadzące do powiązanych tematycznie innych artykułów,
    \item \textit{categorylinks.sql.gz} – zawiera, analogiczny do poprzedniego pliku, spis połączeń między kategoriami oraz między artykułami a kategoriami.
\end{enumerate}

Rozmiary wymienionych plików zależą od wielkości Wikipedii, z której pochodzą. Suma rozmiarów tych trzech plików dla angielskiej Wikipedii wynosi około 10GB, zaś dla Polskiej około 1GB. Istnieje także wiele innych, drobniejszych encyklopedii (w mniej popularnych językach oraz zawierających dane z innych projektów). Ich rozmiary mogą mieścić się w kilku megabajtach. Wielkość danych po dekompresji z formatu \textit{.gz} jest wstanie wzrosnąć nawet dziesięciokrotnie.
\newpage
	\section{Parsowanie informacji}
Po pobraniu i rozpakowaniu rozpoczyna się proces parsowania danych. Każdy z plików \textit{.sql} składa się z definicji struktury tabeli bazy danych oraz listy wpisów w postaci ukazanej na Listing \ref{page_sql}. Parsowanie informacji polega na przejściu po każdej linijce danego pliku, wydostaniu kolejnych wartości następujących po wyrażeniu \textit{VALUES} i zapisaniu tylko tych, które są istotne dla dalszego przetwarzania.

\begin{lstlisting}[language=SQL,frame=single,caption={Fragment pliku enwiki-20191101-page.sql zawierający dane o stronach},label=page_sql]
INSERT INTO `page` VALUES
    (10,0,'AccessibleComputing','',1,0,0.33167112649574004,'20191003224230','20190105021557',854851586,94,'wikitext',NULL),
    (12,0,'Anarchism','',0,0,0.786172332974311,'20191101063615','20191031183024',923631615,104479,'wikitext',NULL),
    (13,0,'AfghanistanHistory','',1,0,0.0621502865684687,'20191029091312','20190618192734',783865149,90,'wikitext',NULL),
    -- \dots
\end{lstlisting}

Plik \textit{page.sql} zawiera nieposortowane informacje o różnych stronach Wikipedii. W celu odróżnienia strony artykułu od strony kategorii (wraz z informacjami o ich identyfikatorach i tytułach) używana jest druga wartość w ciągu pojedynczego wpisu, oznaczająca przestrzeń nazw (parametr \textit{page_namespace}). Według dokumentacji MediaWiki, liczba równa 0 oznacza typową stronę artykułu, a liczba 14 stronę typu kategoria. Na podstawie tego rozróżnienia tworzone są dwa nowe pliki zawierające 2-elementowe krotki, których pierwszym elementem jest identyfikator strony, a drugim jego tytuł.

Pliki \textit{pagelinks.sql} i \textit{categorylinks.sql} są parsowane w podobny sposób. Z każdego wpisu pobierany jest identyfikator artykułu lub kategorii z którego połączenie wychodzi oraz tytuł artykułu lub kategorii do którego połączenie te kieruje. Zanim jednak informacje o połączeniach zostaną zapisane do osobnych plików, potrzebne jest przekształcenie tytułu (drugiego pobieranego parametru) do odpowiadającego identyfikatora strony, tak aby z postaci \textit{ID_strony -> Tytuł_strony} otrzymać postać \textit{ID_strony -> ID_strony}. To pozwoli na zmniejszenie wielkości pliku wynikowego i łatwiejszą do dalszego przetwarzania strukturę. Dodatkowo, dzięki informacji o pochodzeniu odnośnika w pliku \textit{categorylinks.sql} następuje podział zawieranych połączeń na te określające związek między dwoma kategoriami (tworzące strukturę hierarchiczną stron kategorii) oraz związek między artykułem a kategorią (przypisanie artykułu do kategorii).

Po wykonaniu wymienionych przekształceń otrzymywane jest 5 nowych plików (oznaczonych rozszerzeniem \textit{.map}), zawierających dane potrzebne do stworzenia struktury właściwego grafu. Wszystkie te pliki są dodatkowo sortowane alfanumerycznie w celu przyspieszenia kolejnego etapu ich przetwarzania. Wytworzone zostały:

\begin{enumerate}[label=\textbullet]
    \item \textit{page.map} – identyfikatory i tytuły artykułów,
    \item \textit{category.map} – identyfikatory i tytuły kategorii,
    \item \textit{pagelinks.map} – identyfikatory artykułów i odpowiadające im listy identyfikatorów artykułów, do których prowadzą odnośniki znajdujące się w ich treści,
    \item \textit{categorylinksfromcategory.map} – analogicznie wyglądający spis połączeń między kategoriami,
    \item \textit{categorylinksfrompage.map} – analogicznie wyglądający spis połączeń między stronami artykułów a stronami kategorii.
\end{enumerate}

Przykład zastosowanych struktur w plikach zawierających tytuły stron i plikach połączeń ilustruje Rysunek \ref{fig:page-map}. Widoczne dane zostały stworzone na podstawie Wikipedii \textit{simplewiki} w dniu 20 października 2019r. Kolorem niebieskim oznaczony został artykuł o ID 48 zatytułowany "Astronomy". Wśród jego połączeń do innych artykułów znajduje się oznaczony na pomarańczowo artykuł o ID 51. Jest to strona o nazwie "Asteroid". Rysunek \ref{fig:astronomy} to wycinek ekranu prezentujący artykuł "Astronomy" w Wikipedii \textit{simplewiki}. Data wykonania tego zrzutu ekranu to 24 października 2019r. Można zauważyć, że jednym z jego odnośników to faktycznie artykuł "Asteroid". Jest to również siódmy link - licząc od początku treści artykułu - zarówno w pliku \textit{pagelinks.map} jak i na stronie internetowej Wikipedii (zachowana jest ich kolejność).

\img{\chapterPath/img/page_map.png}{Fragment pliku page.map z tytułami (po lewej) oraz pliku pagelinks.map (po prawej)}{page-map}{0.8}
\img{\chapterPath/img/pagelinks_map.png}{Fragment pliku page.map z tytułami (po lewej) oraz pliku pagelinks.map (po prawej)}{pagelinks-map}{0.8}

!!! TODO: Skleić te obraski, aby wyświetlały się obok siebie !!!

\img{\chapterPath/img/astronomy.png}{Zrzut ekranu artykułu zatytułowanego "Astronomy"}{astronomy}{0.8}
	\section{Tworzenie struktury grafu}
\sectionauthor{Jan Kruczyński}
\label{sec:data-files}
Celem tego etapu jest wytworzenie plików, na których operuje aplikacja. Podczas projektowania ich struktury wzięto pod uwagę, że muszą one:
\begin{enumerate}
    \setlength\itemsep{0.2em}
    \item Zawierać w sobie pełną informację na temat struktury skierowanego grafu połączeń pomiędzy kolejnymi węzłami.
    \item Rozróżniać, czy dany węzeł reprezentuje artykuł czy kategorię.
    \item Przechowywać tytuły z Wikipedii każdego węzła.
    \item Przechowywać ID strony na Wikipedii.
    \item Umożliwiać szybkie odnajdywanie interesujących nas danych o konkretnym węźle bez przeszukiwania wszystkich plików za każdym razem, gdy potrzebujemy wydobyć jakąś informację.
    \item Przechowywać dane w skompresowanej formie - bez zbędnych bajtów.
\end{enumerate}

Struktura owych plików powinna dać możliwość funkcjonowania aplikacji bez trzymania wszystkich danych w pamięci tymczasowej. Odczytywanie danych z plików powinno być możliwie najszybsze.

\subsection{Generowanie brakujących danych}
\label{sec:generating-missing-files}

Dane linków zawierają wyłącznie połączenia typu ``z węzła - do innego węzła'', ale nie mają połączeń odwrotnych ``do węzła z innych węzłów''. Aby aplikacja była w stanie zaprezentować pełny skierowany graf połączeń potrzebne jest odwzorowanie odwrotne.
Do finalnego, poprawnego generowania plików potrzebne są dodatkowe pliki:
\begin{itemize}
    \setlength\itemsep{0.2em}
    \item Artykuły, które prowadzą do konkretnego artykułu (odwrotność \codeinline{pagelinks.map})
    \item Kategorie, do których należy dany artykuł (odwrotność categorylinksfrompage.map)
    \item Kategorie, do których należy dana kategoria (odwrotność categorylinksfromcategory.map)
\end{itemize}

Dane, które są potrzebne, nie znajdują się w plikach SQL, lecz dzięki opisanym w sekcji \ref{sec:data-parsing} plikom \codeinline{.map}, istnieje możliwość generacji na ich podstawie brakujących informacji. Aby to zrobić, należy odczytać interesujący plik \codeinline{.map} i linia po linii wypełnić słownik o następującej formie:

\begin{lstlisting}[caption={Słownik przechowujący odwzorowanie odwrotne}, label=lst:reverse-map]
Dictionary<int, List<string>> reverseMap = new Dictionary<int, List<string>>();
\end{lstlisting}

Jako że pliki .map są uporządkowane według ID, aby osiągnąć ten sam efekt, słownik zdefiniowany w listingu \ref{lst:reverse-map} został umieszczony w strukturze \codeinline{SortedDictionary}, a następnie zapisany w pliku o~odpowiedniej nazwie. 

Klucz w słowniku to ID Wikipedii danego artykułu lub kategorii (w zależności od pliku~\codeinline{.map}, który przetwarzamy), typu \codeinline{int}, aby umożliwić łatwe sortowanie. Wartość danego klucza to lista powiązanych połączeń, analogiczna do pliku źródłowego \codeinline{.map}. Jako że nie ma potrzeby rzutowania na~typ numeryczny - odczyt i zapis operuje na typie \codeinline{string} - lista przechowuje właśnie takie wartości. Przykład odwzorowania odwrotnego widać na listingach \ref{lst:page_links} i \ref{lst:rev_page_links}.

\begin{figure}[!h]
\begin{center}
    \begin{minipage}[c]{0.45\linewidth}
        \begin{lstlisting}[frame=single,caption={Przykładowy fragment pliku \lstinline{pagelinks.map}},label=lst:page_links]
1   12,18,20
4   11,12,13
11  12,13,17,20
19  11
33  13,20
41  12,13
\end{lstlisting}
    \end{minipage}
    \hspace{1em}
    \begin{minipage}[c]{0.45\linewidth}
        \begin{lstlisting}[frame=single,caption={Odwzorowanie odwrotne z listingu \ref{lst:page_links} (fragment \lstinline{R\_pagelinks.map})},label=lst:rev_page_links]
11  4,19
12  1,4,11,41
13  4,11,33,41
17  11
18  1
20  1,11,33
\end{lstlisting}
\end{minipage}
\end{center}
\end{figure}
Po zakończeniu tego etapu zostały wygenerowane następujące pliki:
\begin{itemize}
    \setlength\itemsep{0.2em}
    \item \codeinline{R\_pagelinks.map}
    \item \codeinline{R\_categorylinksfrompage.map}
    \item \codeinline{R\_categorylinksfromcategory.map}
\end{itemize}

\subsection{Opis poszczególnych plików}
\label{sec:opis-plikow}
Aby osiągnąć postawione wymagania, utworzono strukturę rozbitą na 5 plików. Wszystkie posiadają tą samą nazwę. Jest nią nazwa wersji Wikipedii, którą opisują (np. ``simplewiki'' lub ``plwiki''). Różnią się rozszerzeniami, gdyż każdy plik posiada inną strukturę.

\paragraph{Plik mapy \codeinline{.wgm}}
Jest to plik instruujący aplikację, na którym miejscu w innych plikach odnajdzie interesujące dane. Każdy węzeł zawiera swój wpis w pliku \codeinline{.map} zajmując dokładnie 12 bajtów o strukturze opisanej w tablicy \ref{tab:structure-mapfile}. Offset informuje, od którego bajtu w danym pliku możemy odczytywać informację o danym węźle.

\tabela{
 \hline
 4 bajty & 4 bajty & 4 bajty \\ [0.5ex] 
 \hline\hline
 Offset w pliku \codeinline{.wgg} & Offset w pliku \codeinline{.wgt} & ID Wikipedii \\\hline
}{|c | c | c|p{0.4\textwidth}| }{Reprezentacja pojedynczego węzła w pliku \lstinline{.wgm}}{structure-mapfile}

W tym miejscu następuje swoiste przekonwertowanie ID Wikipedii danej strony na nowy ID, którym jest numer w kolejności danego węzła w pliku \codeinline{.map}. W aplikacji oraz w pozostałych plikach, gdy znajduje się odniesienie do jakiegoś węzła, użyto nie jego faktycznego ID Wikipedii, ale nowo utworzony identyfikator, rozpoczynający się od zera. Znając go można dokonać mnożenia przez rozmiar każdego wpisu (12) i otrzymać offset w pliku \codeinline{.wgm}.

Podczas konstrukcji plików informacje o danym węźle są jednocześnie umieszczane we wszystkich plikach. Dzięki temu nie ma potrzeby przechowywać danej informującej ile bajtów należy odczytać. Odczytu należy dokonać aż do pozycji offset, który znajduje się w następnym węźle z pliku \codeinline{.wgm}, czyli 12 bajtów dalej.

\paragraph{Plik struktury grafu \codeinline{.wgg}}

Jest to plik zawierający dane połączeń skierowanego grafu. Rozróżniane są połączenia: \#1 od których węzłów można dojść do aktualnego węzła oraz \#2 do których węzłów prowadzi aktualny węzeł. Struktura użyta w tym pliku jest opisana w tablicy \ref{tab:structure-graphfile}.

\tabela{
 \hline
 3 bajty & \#1: 3 bajty * N & \#2: 3 bajty * X \\ [0.5ex] 
 \hline\hline
 N: Ilość połączeń typu \#1 & a$_{1}$, b$_{2}$, c$_{3}$, \dots\, z$_{N}$ & A$_{1}$, B$_{2}$, C$_{3}$, \dots\, Z$_{X}$ \\\hline
}{|c | c | c|p{0.4\textwidth}| }{Reprezentacja pojedynczego węzła w pliku \lstinline{.wgg}}{structure-graphfile}

A$_{1}$, B$_{2}$, C$_{3}$ oraz a$_{1}$, b$_{2}$, c$_{3}$ to nie są ID artykułów Wikipedii, lecz informacje, który z kolei artykuł z pliku \codeinline{.wgm} mamy na myśli. Jako że ilość węzłów nigdy nie przekracza $2^{24}$, 3 bajty wystarczają na~przekazanie tej informacji.

\paragraph{Plik informacyjny \codeinline{.wgi}}

Zawiera wyłącznie jedną, 4-bajtową liczbę typu \codeinline{int} - jest to ilość artykułów, które znajdują się w plikach. Jest to jedyne rozróżnienie dla aplikacji, który węzeł traktować jako artykuł, a który jako kategorię - w pozostałych plikach nie ma pomiędzy nimi rozróżnienia. Do plików najpierw są~zapisywane wszystkie artykuły, dzięki czemu przy pobieraniu danych o węźle, aplikacja oznacza ów~węzeł jako kategorię, gdy jego numer w pliku \codeinline{.wgm} jest większy niż wartość w pliku \codeinline{.wgi}. 

\paragraph{Plik tytułów \codeinline{.wgt}}

Zawiera w sobie zapisane w formacie UTF-8 tytuły wszystkich artykułów i kategorii.

\paragraph{Plik odwzorowań posortowanych tytułów \codeinline{.wgs}}

Aby umożliwić szybkie działanie wyszukiwarki, utworzono oddzielny plik o prostej do przeszukiwania strukturze (tablica \ref{tab:structure-sortfile}). Zawiera on posortowane alfabetycznie wszystkie tytuły - zarówno kategorie jak i artykuły, zakodowane w formacie UTF-8. Aby~ułatwić przeszukiwanie, każdy węzeł jest reprezentowany przez oddzielny wiersz o formacie:

\tabela{
 \hline
 Tytuł węzła & ";" & Numer węzła (od 0) w pliku \codeinline{.wgm} & "\textbackslash n" \\\hline
}{|c | c | c | c|p{0.4\textwidth}| }{Reprezentacja pojedynczego węzła w pliku \lstinline{.wgs}}{structure-sortfile}


\subsection{Metoda generowania plików}

Program generujący pliki \codeinline{.wg}\textit{X} został napisany w języku C\#.
Program zawiera kilka pomocniczych struktur ułatwiających konstrukcję wynikowych plików:

\begin{lstlisting}[caption={Pomocnicze struktury dla programu generującego pliki dla aplikacji}, label=lst:graph-object]
Dictionary<int, int> pageMap;%*\label{line:pageMap}*)
Dictionary<int, int> categoryMap;%*\label{line:categoryMap}*)
List<string, int> sortedTitles;%*\label{line:sortedTitles}*)

public class GraphObject {
    public bool isArticle;%*\label{line:go-article}*)
    public int id;%*\label{line:go-id}*)
    public string title;%*\label{line:go-title}*)
    public int offsetTitle;%*\label{line:go-offsetTitle}*)
    public int offsetGraph;%*\label{line:go-offsetGraph}*)
    public int order;%*\label{line:go-order}*)
}
\end{lstlisting}

Na listingu \ref{lst:graph-object} linijki \ref{line:pageMap} i \ref{line:categoryMap} to mapy przechowujące odwzorowanie ID z Wikipedii (\ref{line:go-id}) (który reprezentuje dany węzeł w plikach \codeinline{.map}) na numer w kolejności węzła w pliku \codeinline{.wgm} (nowo utworzony ID (\ref{line:go-order})). Jako że kategorie oraz artykuły mogą posiadać taki sam ID Wikipedii, potrzebne na~to~są~dwie oddzielne struktury. Konstrukcja w wierszu \ref{line:sortedTitles} stanowi listę przechowującą krotki tytułu węzła i~odwzorowania ID do późniejszej generacji posortowanych tytułów (wartości z linii \ref{line:go-title} i~\ref{line:go-order})

Każdy węzeł podczas przetwarzania jest traktowany jako \codeinline{GraphObject} i zawiera w sobie informacje o tytule na Wikipedii (\ref{line:go-title}), liczbie informującej od którego bajtu, w pliku tytułów \codeinline{.wgt} (\ref{line:go-offsetTitle}) oraz strukturze grafu \codeinline{.wgg} (\ref{line:go-offsetGraph}) zostanie zapisana ta informacja oraz o tym, czy jest artykułem czy~kategorią~(\ref{line:go-article}).

\paragraph{Operacje przygotowujące}
Zanim program rozpocznie generację właściwych plików \codeinline{.wg}\textit{X}, potrzebuje dokonać generacji. W pierwszej kolejności program wywołuje metodę generującą odwzorowania odwrotne dla każdego z wymagających tego trzech plików, zgodnie z metodą opisaną w rozdziale \ref{sec:generating-missing-files}.

Następnie program odczytuje tytuły wszystkich artykułów i linia po linii zapełnia mapę \codeinline{pageMap} (\ref{line:pageMap}) oraz \codeinline{sortedTitles} (\ref{line:sortedTitles}). Po tym etapie odczytuje tytuły kategorii i wypełnia categoryMap oraz dalej uzupełnia listę sortedTitles w analogiczny sposób. Iteracja jest jednorazowa po wszystkich tytułach, więc złożoność obliczeniowa tego etapu to $O(n)$.

Dzięki zapisaniu mapy odwzorowań tytułów (\ref{line:sortedTitles}) w krotkach, sortowanie możliwe jest do realizacji metodą \codeinline{Array.Sort()} z własną implementacją porównania elementów \codeinline{IComparer<T>}. Metoda sortująca dostarczona jest przez samą technologię .NET, a algorytm sortujący to implementacja algorytmu QuickSort, który ma średnią złożoność obliczeniową $O(n\log n)$.

Następnie program odczytuje wszystkie elementy z posortowanej listy i zapisuje je pojedynczo do~pliku \codeinline{.wgs} dla tytułów artykułów oraz \codeinline{sortedCategoryTitles.map} dla kategorii. Następuje iteracja po~każdym elemencie listy, co oznacza złożoność obliczeniową $O(n)$. Po dokonaniu zapisu mapy tytułów nie są już potrzebne w dalszej części programu, więc następuje ich usunięcie z pamięci.

Uznając ilość artykułów za $n$, a ilość kategorii za $m$, łączna złożoność obliczeniowa algorytmów etapu operacji, przygotowujących generację plików, została przedstawiona we wzorze \ref{eq:initial_operations}.

\begin{equation}
O(2n) + O(n\log n) + O(2m) + O(m \log m) = O(n\log n + m\log m)
\label{eq:initial_operations}
%\caption{Złożoność obliczeniowa etapu przygotowującego generację plików}
\end{equation}

\paragraph{Generacja plików Wikigraphu}
Kolejnym etapem jest już faktyczne zapisanie danych o węźle do~wszystkich plików. Liczba artykułów jest już znana - wystarczy policzyć elementy w mapie \codeinline{pageMap} (linia~\ref{line:pageMap} w~listingu \ref{lst:graph-object}) - i zapisać ją do pliku \codeinline{.wgi}.

Aby zachować taką samą kolejność węzłów, ponownie dokonany jest odczyt pliku tytułów dla~artykułów i~dla~każdego węzła parsowane są informacje do struktury reprezentującej dany element w~programie (listing \ref{lst:graph-object}). Program jest napisany w postaci klasy zawierającej wszystkie metody zapisu i~odczytu danych, co umożliwia łatwe przekazywanie zmiennych pomiędzy metodami, bez konieczności przekazywania wszystkiego w argumentach.

Plik z tytułami zawiera wszystkie interesujące nas elementy. Nie istnieje element w innych plikach, który nie posiada swojej reprezentacji w pliku z tytułami (element z pliku z tytułami może natomiast nie~zawierać wpisu w plikach połączeń). Program przechowuje w głównej klasie numery linii dla każdego pliku z połączeniami, które powinniśmy aktualnie odczytać.

\begin{enumerate}
\item Podczas przetwarzania artykułu odczytujemy pliki
\begin{itemize}[label=\textbullet]
    \item  \codeinline{pagelinks.map}
    \item  \codeinline{R\_pagelinks.map}
    \item  \codeinline{categorylinksfrompage.map}
\end{itemize}
\item Podczas przetwarzania kategorii odczytujemy pliki
\begin{itemize}[label=\textbullet]
    \item  \codeinline{R\_categorylinksfromcategory.map}
    \item  \codeinline{categorylinksfromcategory.map}
    \item  \codeinline{R\_categorylinksfrompage.map}
\end{itemize}
\end{enumerate}

Przy odczytaniu wartości ID w pliku tytułów, odczytywane są także pojedyncze linie w powyższych plikach \codeinline{.map}. Jeżeli ID w pliku z połączeniami odpowiada ID w pliku tytułów, te połączenia są~uwzględniane do aktualnego węzła, a licznik odczytu linii dla danego pliku jest zwiększany. Jeżeli w~pliku połączeń nie znaleziono wartości ID z pliku tytułów, dany artykuł bądź kategoria nie zawiera w~tym pliku \codeinline{.map} żadnych połączeń, a licznik dla tego pliku nie jest zwiększany.

Posiadając wszystkie informacje o danym węźle, używając klasy \codeinline{BinaryWriter} dane zostają dopisane do plików \codeinline{.wgg} i \codeinline{.wgt} zgodnie z ich strukturą opisaną w rozdziale \ref{sec:opis-plikow}. Ilość dopisanych bajtów do każdego z tych plików jest dodawana do odpowiednich liczników, a ich wartości zostają wraz z ID z Wikipedii dopisane do pliku \codeinline{.wgm}. Całość jest powtarzana dla każdego z artykułów, a następnie dla każdej z kategorii, zgodnie z plikiem z tytułami.

Ponieważ kategorie znajdują się na końcu pliku \codeinline{.wgm}, w momencie odczytywania ich kolejności w pliku \codeinline{.wgm} z mapy odwzorowań (listing \ref{lst:graph-object} linia \ref{line:categoryMap}) do tej wartości dodawana jest ilość artykułów, czyli liczba zapisana w pliku \codeinline{.wgi}. Przykładowo pierwsza kategoria będzie po ostatnim artykule, więc jeżeli jej ID oryginalnie wynosiło 0 (jako pierwsza w kolejności), jej nowy ID z mapy odwzorowań będzie równe ilości artykułów.

Złożoność obliczeniowa tego etapu, ponieważ następuje tutaj wyłącznie pojedyncza iteracja po~każdym artykule i kategorii, wynosi $O(n+m)$, gdzie $n$ to ilość artykułów, a $m$ to ilość kategorii. Główna czasochłonność pochodzi z czasu otwierania i zamykania strumieni dostępu do plików oraz odczytywania danych.

	\section{Narzędzie WikiGraph Parser}
\sectionauthor{Mikołaj Mirko}
\label{sec:parser-tool}
Proces pozyskiwania danych opisany w podrozdziałach \ref{sec:data-source} - \ref{sec:data-files} został zautomatyzowany i zaimplementowany w postaci dodatkowego, pomocniczego programu. Aplikacja WikiGraph Parser ma na celu usprawnienie pracy z pozyskiwanymi danymi oraz zmniejszyć ryzyko wystąpienia nieprawidłowości. Aplikacja została napisana w języku C\# (z platformą docelową .NET Framework 4.6.1) i posiada interfejs graficzny stworzony za pomocą frameworka UI WPF.

Interfejs programu został zaprojektowany z myślą o 10 heurystykach Nielsena \cite{Heuristics}. Zastosowano estetyczny i minimalistyczny wygląd wykorzystujący typowe standardy aplikacji okienkowych. W wielu miejscach użytkownik jest informowany o stanie programu za pomocą etykiet oraz symboli graficznych. Walidacja pozwala uniknąć przewidywalnych błędów, a w razie wystąpienia nieoczekiwanych problemów pojawiają się opisowe okna dialogowe. Pomoc zawarta jest w postaci tytułów sekcji i podpowiedzi, a dokumentacja (instrukcja obsługi) opisana jest w rozdziale \ref{chap:config}. Konfiguracja i uruchamianie.

Użytkownik obsługujący WikiGraph Parser ma możliwość wyboru interesującego go źródła informacji oraz określenie ścieżki aplikacji, wykorzystującej stworzone zasoby (Rysunek \ref{fig:parser1}). Ograniczając go do wyboru tylko tych dwóch parametrów tworzony jest prosty interfejs, zachowując jednocześnie swobodę i kontrolę użytkowania. Giles Colborne w swojej książce ``Prostota i użyteczność'' \cite{ProstotaUzytecznosc} opisuje zasadę zachowania złożoności, według której zabiegi usuwania i ukrywania funkcjonalności przed użytkownikiem sprowadzają interfejs do jego minimalnego poziomu złożoności.

\begin{center}
	\hyphenblockcquote{polish}{ProstotaUzytecznosc}{
        Cała sztuka projektowania prostych rozwiązań polega na przenoszeniu złożoności w odpowiednie miejsce, tak aby korzystanie z samego narzędzia było łatwe.
	}
\end{center}

\img{\chapterPath/img/parser1.png}{Ekran konfiguracji parametrów WikiGraph Parser-a}{parser1}{0.8}

\img{\chapterPath/img/parser2.png}{Ekran statusu postępu przetwarzania danych}{parser2}{0.8}

Przetwarzanie danych jest rozpoczynane po naciśnięciu przycisku \textit{Start}. Na ekranie z informacją o aktualnym postępie (Rysunek \ref{fig:parser2}) można wyróżnić pięć kroków. Są to:

\begin{enumerate}[label=\textbullet]
    \item \textit{Downloading dumps}

    Na podstawie wybranego zrzutu bazy pobierany jest odpowiedni zestaw plików (szczegółowo opisany w podrozdziale \ref{sec:data-download}). Do tego zadania wykorzystana jest klasa WebClient.
    
    \item \textit{Decompressing dumps} \linebreak
    
    Dekompresja pobranych plików odbywa się za pomocą klasy GzipStream. Po tym kroku pliki \textit{.sql} gotowe są do przetwarzania.
    
    \item \textit{Reading data from dumps} \linebreak
    
    Podczas tego kroku tworzone są pliki z rozszerzeniem \textit{.map}, zawierające tylko te dane, które są wykorzystywane do dalszego przetwarzania. Na tym etapie odbywa się również sortowanie danych – użyty jest pakiet ``Sortiously''\cite{Github:Sortiously}.
    
    \item \textit{Generating WikiGraph files} \linebreak
    
    Ten krok przetwarza przerobione i uporządkowane dane do plików grafu o strukturze binarnej, czytanych przez aplikację główną (podrozdział \ref{sec:data-files}). Wykorzystane są m.in. klasy takie jak BinaryWriter oraz BinaryConverter.
    
    \item \textit{Saving files and cleaning up} \linebreak	
    
    Po zakończeniu poprzednich etapów, niepotrzebne, tymczasowe pliki zostają usunięte z dysku. We wskazanym folderze powinna pojawić się struktura folderów wspierająca możliwość posiadania wielu wersji na raz.

\end{enumerate}
\end{chapter}
