\section{Źródło danych}
Internetowa encyklopedia Wikipedia to jeden z projektów organizacji Wikimedia Foundation. Projekt ten ma na celu gromadzenie, porządkowanie i weryfikowanie otwartych danych tworzonych przez społeczność wolontariuszy (często nazywanych „Wikipedystami”). Według Podstawowego Rankingu Międzyjęzykowego \cite{Wikipedia:PodstawowyRanking} liczba artykułów w angielskiej wersji językowej na dzień 1 listopada 2019r. wynosi prawie 6 milionów, a co miesiąc dochodzi blisko 20 tysięcy nowych pojęć. Przechowywanie takiej ilości danych (uwzględniając całą treść i media artykułu, znajdujące się w nim powiązania wewnętrze i zewnętrzne oraz historię jego edycji) jest nie lada wyzwaniem.

Wikimedia Foundation prowadzi również inne inicjatywy, takie jak m.in. Wikibooks (zbiór książek i podręczników), Wikinews (dziennik wydarzeń) i Wikiquote (kolekcja rozmaitych cytatów) – każda z nich posiadająca wiele wersji językowych. Wszystkie oferowane przez fundację serwisy opierają się na oprogramowaniu MediaWiki. Odpowiada ono za ogólną strukturę strony typu wiki, oferując jednocześnie wiele dodatkowych mechanizmów ułatwiających pracę z dużą ilością danych. Skonfigurowane zewnętrzne API pozwala na dostęp do danych innym oprogramowaniem, użycie szablonów stron ułatwia oddzielenie warstwy wizualnej od samych danych, a moduł archiwizacji odpowiada za tworzenie kopii zapasowych baz danych.

Ostatnia z przytoczonych funkcjonalności odgrywa dużą rolę w sposobie zdobycia danych do aplikacji. Kolejny projekt fundacji, o którym trzeba wspomnieć to Meta-Wiki. Odpowiedzialny jest on za koordynację wszystkich pozostałych projektów. Zawiera bogatą dokumentację, historię zmian i aktualizacji oraz raporty aktywności. Udostępnia również publiczne zrzuty baz danych, zawierające część informacji z każdej dostępnej wersji językowej każdego projektu. Wykonywane są one z częstotliwością około 1 raz na 2 tygodnie i używają wspominanego modułu archiwizacji (choć nie są typową kopią zapasową całego serwisu).

Wśród oferowanych danych w zrzutach znaleźć można znaleźć przede wszystkim informacje o stronach (czyli artykułach, kategoriach, szablonach, przestrzeniach nazw i kilku innych), historii ich edycji oraz wewnętrznych połączeniach między stronami. Oprócz tego dostępne są różnego rodzaju listy, statystyki i metadane. Niektóre z nich dostępne są w formatach takich jak \textit{.txt}, \textit{.json} i \textit{.xml}, ale większość z nich zapisana jest w postaci plików \textit{.sql} zawierających definicję tabeli bazy danych oraz ciąg wpisów z danymi. Wszystkie udostępniane pliki są dodatkowo zarchiwizowane przy pomocy programu GZIP.